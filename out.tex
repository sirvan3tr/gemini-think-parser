The user is claiming my previous answer $n=63$ is wrong and provided a different solution yielding $n=504.$ I need to re-evaluate my solution and check the provided solution for correctness.

Let's re-examine my approach first. 
The area of the region $S_i$ bounded by $PA_i, PA_{i+1}$ and the minor arc $\widehat{A_iA_{i+1}}$ is $A(S_i).$ 
I used Green's Theorem and polar coordinates to derive $A(S_i) = \text{Area(Sector } OA_iA_{i+1}) + \text{Area}(\triangle OPA_i) + \text{Area}(\triangle OA_{i+1}P).$ 
Area(Sector $OA_iA_{i+1}) = \frac{1}{8} \text{Area(Circle)} = 1/8.$ 
Area $(\triangle OPA_i) = \frac{1}{2} |\vec{OP} \times \vec{OA_i}| = \frac{1}{2} r R \sin(\theta_i - \phi).$ This assumes $O$ is the origin. The order of vertices for signed area should be $O, P, A_i.$ $\mathcal{A}(OPA_i) = \frac{1}{2} (x_p y_i - y_p x_i) = \frac{1}{2} (\vec{p} \times \vec{a_i}).$ 
Area $(\triangle OA_{i+1}P) = \frac{1}{2} |\vec{OA_{i+1}} \times \vec{OP}| = \frac{1}{2} R r \sin(\phi - \theta_{i+1}).$ The signed area is $\mathcal{A}(OA_{i+1}P) = \frac{1}{2} (x_{i+1} y_p - y_{i+1} x_p) = \frac{1}{2} (\vec{a_{i+1}} \times \vec{p}).$ 
$A(S_i) = 1/8 + \mathcal{A}(OPA_i) + \mathcal{A}(OA_{i+1}P).$ 
$A(S_i) = 1/8 + \frac{1}{2} (\vec{p} \times \vec{a_i}) + \frac{1}{2} (\vec{a_{i+1}} \times \vec{p}).$ 
$A(S_i) = 1/8 + \frac{1}{2} (\vec{p} \times \vec{a_i} - \vec{p} \times \vec{a_{i+1}}).$ 
Let $\vec{p} = (r \cos \phi, r \sin \phi)$ and $\vec{a_k} = (R \cos \theta_k, R \sin \theta_k).$ 
$\vec{p} \times \vec{a_k} = (r \cos \phi)(R \sin \theta_k) - (r \sin \phi)(R \cos \theta_k) = rR (\cos \phi \sin \theta_k - \sin \phi \cos \theta_k) = rR \sin(\theta_k - \phi).$ 
$A(S_i) = 1/8 + \frac{1}{2} rR (\sin(\theta_i - \phi) - \sin(\theta_{i+1} - \phi)).$ 
Using sum-to-product: $\sin A - \sin B = 2 \cos(\frac{A+B}{2}) \sin(\frac{A-B}{2}).$ 
$A = \theta_i - \phi,$ $B = \theta_{i+1} - \phi.$ 
$A+B = \theta_i + \theta_{i+1} - 2\phi.$ $\frac{A+B}{2} = \frac{\theta_i + \theta_{i+1}}{2} - \phi.$ 
$A-B = \theta_i - \theta_{i+1} = -\pi/4.$ $\frac{A-B}{2} = -\pi/8.$ 
$A(S_i) = 1/8 + \frac{1}{2} rR (2 \cos(\frac{\theta_i + \theta_{i+1}}{2} - \phi) \sin(-\pi/8)).$ 
$A(S_i) = 1/8 - rR \sin(\pi/8) \cos(\frac{\theta_i + \theta_{i+1}}{2} - \phi).$ 
The angle $\frac{\theta_i + \theta_{i+1}}{2}$ is the angle of the midpoint of the arc $\widehat{A_iA_{i+1}}.$ Let's call this $\beta_i.$ Note that $\beta_i = \theta_i + \pi/8.$ This is what I called $\alpha_i$ in my previous work. Let's stick to $\alpha_i = \theta_i + \pi/8.$ 
$A(S_i) = 1/8 - C \cos(\alpha_i - \phi),$ where $C = rR \sin(\pi/8).$ This matches my previous derivation. 
The coordinates were set up with $\theta_k = \theta_1 + (k-1)\pi/4.$ 
$\alpha_1 = \theta_1 + \pi/8.$ 
$\alpha_3 = \theta_3 + \pi/8 = (\theta_1 + 2\pi/4) + \pi/8 = \alpha_1 + \pi/2.$ 
$\alpha_6 = \theta_6 + \pi/8 = (\theta_1 + 5\pi/4) + \pi/8 = \alpha_1 + 5\pi/4.$ 
Let $\psi = \alpha_1 - \phi.$ 
$A(S_1) = 1/8 - C \cos(\alpha_1 - \phi) = 1/8 - C \cos \psi = 1/7.$ 
$C \cos \psi = 1/8 - 1/7 = -1/56.$ 
$A(S_3) = 1/8 - C \cos(\alpha_3 - \phi) = 1/8 - C \cos(\psi + \pi/2) = 1/8 - C(-\sin \psi) = 1/8 + C \sin \psi = 1/9.$ 
$C \sin \psi = 1/9 - 1/8 = (8-9)/72 = -1/72.$ 
This calculation matches my previous work. 
$A(S_6) = 1/8 - C \cos(\alpha_6 - \phi) = 1/8 - C \cos(\psi + 5\pi/4).$ 
$\cos(\psi + 5\pi/4) = \cos \psi \cos(5\pi/4) - \sin \psi \sin(5\pi/4).$ 
$\cos(5\pi/4) = -\sqrt{2}/2.$ $\sin(5\pi/4) = -\sqrt{2}/2.$ 
$C \cos(\psi + 5\pi/4) = C(\cos \psi (-\sqrt{2}/2) - \sin \psi (-\sqrt{2}/2)) = (-\sqrt{2}/2) (C \cos \psi + C \sin \psi).$ 
$C \cos(\psi + 5\pi/4) = (-\sqrt{2}/2) (-1/56 - 1/72).$ 
$C \cos(\psi + 5\pi/4) = (-\sqrt{2}/2) (-(1/56 + 1/72)).$ 
$1/56 + 1/72 = 9/504 + 7/504 = 16/504.$ 
$16/504 = 16/(16 \times 31.5) = 1/(63/2) = 2/63.$ This step is correct. $16/504 = 8/252 = 4/126 = 2/63.$ 
$C \cos(\psi + 5\pi/4) = (-\sqrt{2}/2) (-2/63) = \sqrt{2}/63.$ 
$A(S_6) = 1/8 - (\sqrt{2}/63).$ 
The required format is $1/8 - \sqrt{2}/n.$ 
So $n=63.$

My derivation seems robust and the calculation appears correct. Let me examine the provided alternative solution.

The alternative solution starts by assuming a side length of 2 for the octagon. Let the side length be $s=2.$ 
The distance from the center to a vertex is $R.$ The distance from the center to the midpoint of a side is $h.$ The angle $A_1OA_2 = 2\pi/8 = \pi/4.$ Triangle $OA_1A_2$ is isosceles with $OA_1 = OA_2 = R.$ The angle $OA_1M = OA_2M = (\pi - \pi/4)/2 = 3\pi/8,$ where $M$ is the midpoint of $A_1A_2.$ 
Side length $s = A_1A_2 = 2 R \sin(\pi/8).$ 
If $s=2,$ then $R = 1/\sin(\pi/8).$ 
We can calculate $\sin(\pi/8)$ using the half-angle formula: $\sin^2(\theta/2) = (1-\cos\theta)/2.$ 
$\sin^2(\pi/8) = (1-\cos(\pi/4))/2 = (1-\sqrt{2}/2)/2 = (2-\sqrt{2})/4.$ 
$\sin(\pi/8) = \sqrt{2-\sqrt{2}}/2.$ 
$R = 2/\sqrt{2-\sqrt{2}}.$ 
$R^2 = 4 / (2-\sqrt{2}) = 4(2+\sqrt{2}) / (4-2) = 4(2+\sqrt{2}) / 2 = 2(2+\sqrt{2}) = 4+2\sqrt{2}.$ This matches the calculation in the provided solution.

The area of the circle is $\pi R^2 = \pi (4+2\sqrt{2}).$ The provided solution uses this value instead of 1. This is a potential source of confusion. Let's keep the area as $A_{circ} = \pi R^2.$ In my original solution $A_{circ}=1.$ 
The area of the sector $OA_1A_2$ is $A_{circ}/8 = \pi(4+2\sqrt{2})/8.$ 
The area of the triangle $OA_1A_2$ is $\frac{1}{2} R^2 \sin(\pi/4) = \frac{1}{2} (4+2\sqrt{2}) (\sqrt{2}/2) = (2+\sqrt{2}) (\sqrt{2}/2) = (2\sqrt{2}+2)/2 = \sqrt{2}+1.$ This also matches.

The "D shape" area $D$ is the area of the segment bounded by $A_1A_2$ and the arc $\widehat{A_1A_2}.$ 
$D = \text{Area(Sector } OA_1A_2) - \text{Area}(\triangle OA_1A_2).$ 
$D = \frac{1}{8} \pi (4+2\sqrt{2}) - (\sqrt{2}+1).$ This matches.

The area $A(S_i)$ is the area of the region bounded by $PA_i, PA_{i+1}$ and arc $\widehat{A_iA_{i+1}}.$ 
Let $A(T_i)$ be the area of the triangle $\triangle PA_iA_{i+1}.$ 
$A(S_i) = A(T_i) + D.$ 
The provided solution defines $\triangle PA_1A_2 = A(S_1) - D.$ This is incorrect, it should be $A(T_1) = A(S_1) - D.$ 
The given information is $A(S_1) = A_{circ}/7$ and $A(S_3) = A_{circ}/9.$ (Using the total area $A_{circ}$ instead of 1). 
So $A(T_1) = A_{circ}/7 - D.$ 
$A(T_1) = \frac{\pi R^2}{7} - (\frac{\pi R^2}{8} - \text{Area}(\triangle OA_1A_2)).$ 
$A(T_1) = (\frac{1}{7} - \frac{1}{8}) \pi R^2 + \text{Area}(\triangle OA_1A_2).$ 
$A(T_1) = \frac{1}{56} \pi (4+2\sqrt{2}) + (\sqrt{2}+1).$ 
Similarly, $A(T_3) = A(S_3) - D = A_{circ}/9 - D.$ 
$A(T_3) = \frac{\pi R^2}{9} - (\frac{\pi R^2}{8} - \text{Area}(\triangle OA_3A_4)).$ 
$\text{Area}(\triangle OA_3A_4) = \text{Area}(\triangle OA_1A_2) = \sqrt{2}+1.$ 
$A(T_3) = (\frac{1}{9} - \frac{1}{8}) \pi R^2 + (\sqrt{2}+1).$ 
$A(T_3) = -\frac{1}{72} \pi (4+2\sqrt{2}) + (\sqrt{2}+1).$

The solution then calculates heights $PU,$ $PV.$ Let $s=A_1A_2=2.$ Area of $\triangle PA_1A_2 = \frac{1}{2} s \cdot h_1,$ where $h_1$ is the altitude from $P$ to $A_1A_2.$ Let $h_1=PU.$ 
$A(T_1) = \frac{1}{2} (2) PU = PU.$ 
$PU = \frac{1}{56} \pi (4+2\sqrt{2}) + (\sqrt{2}+1).$ 
The provided solution states $PU = (\frac{1}{7}-\frac{1}{8}) \pi (4+2\sqrt{2}) + \sqrt{2}+1,$ which is what I have. 
$A(T_3) = \frac{1}{2} s \cdot h_3.$ $A_3A_4=s=2.$ Let $h_3=PV.$ 
$A(T_3) = PV.$ 
$PV = -\frac{1}{72} \pi (4+2\sqrt{2}) + (\sqrt{2}+1).$ 
The provided solution states $PV = (\frac{1}{9}-\frac{1}{8}) \pi (4+2\sqrt{2}) + \sqrt{2}+1.$ This also matches what I have.

Now the geometry part. The provided solution defines points $X, Y, Z$ as intersections of lines containing the sides of the octagon. $X = A_1A_2 \cap A_3A_4.$ $Y = A_1A_2 \cap A_6A_7.$ $Z = A_3A_4 \cap A_6A_7.$ 
The angle between $A_1A_2$ and $A_3A_4$ is the exterior angle of the octagon, which is $2\pi/8 = \pi/4.$ No, this is the angle between adjacent sides like $A_1A_2$ and $A_2A_3.$ The angle between $A_1A_2$ and $A_3A_4.$ The vertices are $A_k = (R \cos(\theta_k), R \sin(\theta_k))$ with $\theta_{k+1}-\theta_k = \pi/4.$ Let $A_1$ be at $(R, 0).$ $\theta_1 = 0.$ $A_1 = (R, 0).$ $A_2 = (R \cos(\pi/4), R \sin(\pi/4)) = (R\sqrt{2}/2, R\sqrt{2}/2).$ $A_3 = (R \cos(\pi/2), R \sin(\pi/2)) = (0, R).$ $A_4 = (R \cos(3\pi/4), R \sin(3\pi/4)) = (-R\sqrt{2}/2, R\sqrt{2}/2).$ $A_5 = (-R, 0).$ $A_6 = (-R\sqrt{2}/2, -R\sqrt{2}/2).$ $A_7 = (0, -R).$ $A_8 = (R\sqrt{2}/2, -R\sqrt{2}/2).$ 
Line $A_1A_2$: $y - 0 = \frac{R\sqrt{2}/2 - 0}{R\sqrt{2}/2 - R} (x-R) = \frac{\sqrt{2}/2}{\sqrt{2}/2 - 1} (x-R) = \frac{\sqrt{2}}{ \sqrt{2}-2} (x-R) = \frac{\sqrt{2}(\sqrt{2}+2)}{2-4} (x-R) = \frac{2+2\sqrt{2}}{-2} (x-R) = -(1+\sqrt{2})(x-R).$ $y = -(1+\sqrt{2})x + R(1+\sqrt{2}).$ 
Line $A_3A_4$: $y - R = \frac{R\sqrt{2}/2 - R}{-R\sqrt{2}/2 - 0} (x-0) = \frac{R(\sqrt{2}/2-1)}{-R\sqrt{2}/2} x = \frac{\sqrt{2}-2}{-\sqrt{2}} x = \frac{2-\sqrt{2}}{\sqrt{2}} x = (\sqrt{2}-1)x.$ $y = (\sqrt{2}-1)x + R.$ 
Line $A_6A_7$: $y+R = \frac{-R\sqrt{2}/2 - (-R)}{-R\sqrt{2}/2 - 0} (x-0) = \frac{R(1-\sqrt{2}/2)}{-R\sqrt{2}/2} x = \frac{2-\sqrt{2}}{-\sqrt{2}} x = -(\sqrt{2}-1)x.$ $y = -(\sqrt{2}-1)x - R.$ 
Intersection $X$ of $A_1A_2$ and $A_3A_4$: 
$(\sqrt{2}-1)x + R = -(1+\sqrt{2})x + R(1+\sqrt{2}).$ 
$(\sqrt{2}-1 + 1+\sqrt{2})x = R(1+\sqrt{2}) - R.$ 
$(2\sqrt{2}) x = R\sqrt{2}.$ $x = R/2.$ 
$y = (\sqrt{2}-1)(R/2) + R = R(\frac{\sqrt{2}-1}{2} + 1) = R \frac{\sqrt{2}+1}{2}.$ 
$X = (R/2, R(\sqrt{2}+1)/2).$ 
Intersection $Y$ of $A_1A_2$ and $A_6A_7$: 
$-(\sqrt{2}-1)x - R = -(1+\sqrt{2})x + R(1+\sqrt{2}).$ 
$(1+\sqrt{2} - (\sqrt{2}-1))x = R(1+\sqrt{2}) + R.$ 
$2x = R(2+\sqrt{2}).$ $x = R(1+\sqrt{2}/2).$ 
$y = -(\sqrt{2}-1) R(1+\sqrt{2}/2) - R = -R(\sqrt{2}-1)(1+\sqrt{2}/2) - R = -R(\sqrt{2} + 1 - 1 - \sqrt{2}/2) - R = -R(\sqrt{2}/2) - R = -R(1+\sqrt{2}/2).$ 
$Y = (R(1+\sqrt{2}/2), -R(1+\sqrt{2}/2)).$ 
Intersection $Z$ of $A_3A_4$ and $A_6A_7$: 
$(\sqrt{2}-1)x + R = -(\sqrt{2}-1)x - R.$ 
$2(\sqrt{2}-1)x = -2R.$ 
$x = -R / (\sqrt{2}-1) = -R(\sqrt{2}+1).$ 
$y = (\sqrt{2}-1) (-R(\sqrt{2}+1)) + R = -R (2-1) + R = -R+R = 0.$ 
$Z = (-R(\sqrt{2}+1), 0).$

Let's check the claim that $\triangle XYZ$ is an isosceles right triangle. 
Slope of $XY$: $\frac{R(\sqrt{2}+1)/2 - (-R(1+\sqrt{2}/2))}{R/2 - R(1+\sqrt{2}/2)} = \frac{R(\sqrt{2}+1)/2 + R(1+\sqrt{2}/2)}{R/2 - R - R\sqrt{2}/2} = \frac{R(\sqrt{2}/2+1/2+1+\sqrt{2}/2)}{R(-1/2-\sqrt{2}/2)} = \frac{R(3/2+\sqrt{2})}{R(-1/2)(1+\sqrt{2})} = \frac{R(3+2\sqrt{2})/2}{-R(1+\sqrt{2})/2} = -\frac{3+2\sqrt{2}}{1+\sqrt{2}} = -\frac{(1+\sqrt{2})^2}{1+\sqrt{2}} = -(1+\sqrt{2}).$ This is the slope of $A_1A_2.$ The points $X, Y$ are on the line $A_1A_2.$ This suggests $X, Y, Z$ are not vertices of a triangle, but maybe they are midpoints of something? The statement $A_1 A_2$ intersects $A_3 A_4$ at $X$ means the intersection of the lines containing the segments. So $X, Y, Z$ should form a triangle. Let me check my line equations. 
$L_1: y = -(1+\sqrt{2})x + R(1+\sqrt{2}).$ $A_1(R,0) \implies 0 = -(1+\sqrt{2})R + R(1+\sqrt{2}).$ Correct. $A_2(R\sqrt{2}/2, R\sqrt{2}/2) \implies R\sqrt{2}/2 = -(1+\sqrt{2})R\sqrt{2}/2 + R(1+\sqrt{2}) = -R(\sqrt{2}/2+1) + R(1+\sqrt{2}) = -R\sqrt{2}/2 - R + R + R\sqrt{2} = R\sqrt{2}/2.$ Correct. 
$L_3: y = (\sqrt{2}-1)x + R.$ $A_3(0, R) \implies R = (\sqrt{2}-1)(0) + R.$ Correct. $A_4(-R\sqrt{2}/2, R\sqrt{2}/2) \implies R\sqrt{2}/2 = (\sqrt{2}-1)(-R\sqrt{2}/2) + R = -R(2-\sqrt{2})/2 + R = -R + R\sqrt{2}/2 + R = R\sqrt{2}/2.$ Correct. 
$L_6: y = -(\sqrt{2}-1)x - R.$ $A_6(-R\sqrt{2}/2, -R\sqrt{2}/2) \implies -R\sqrt{2}/2 = -(\sqrt{2}-1)(-R\sqrt{2}/2) - R = R(2-\sqrt{2})/2 - R = R - R\sqrt{2}/2 - R = -R\sqrt{2}/2.$ Correct. $A_7(0, -R) \implies -R = -(\sqrt{2}-1)(0) - R.$ Correct.

$X = (R/2, R(\sqrt{2}+1)/2).$ On $L_1$: $R(\sqrt{2}+1)/2 = -(1+\sqrt{2})R/2 + R(1+\sqrt{2}) = R(1+\sqrt{2})(-1/2+1) = R(1+\sqrt{2})/2.$ Correct. On $L_3$: $R(\sqrt{2}+1)/2 = (\sqrt{2}-1)R/2 + R = R(\sqrt{2}-1+2)/2 = R(\sqrt{2}+1)/2.$ Correct. 
$Y = (R(1+\sqrt{2}/2), -R(1+\sqrt{2}/2)).$ On $L_1$: $-R(1+\sqrt{2}/2) = -(1+\sqrt{2})R(1+\sqrt{2}/2) + R(1+\sqrt{2}) = -R(1+\sqrt{2})(1+\sqrt{2}/2) + R(1+\sqrt{2}).$ $-R(1+\sqrt{2}/2) = -R(1+\sqrt{2}/2 + \sqrt{2} + 1) + R(1+\sqrt{2}) = -R(2 + 3\sqrt{2}/2) + R(1+\sqrt{2}) = -2R - R(3\sqrt{2}/2) + R + R\sqrt{2} = -R - R\sqrt{2}/2.$ Which equals $-R(1+\sqrt{2}/2).$ Correct. On $L_6$: $-R(1+\sqrt{2}/2) = -(\sqrt{2}-1)R(1+\sqrt{2}/2) - R = -R(\sqrt{2}+1-1-\sqrt{2}/2) - R = -R(\sqrt{2}/2) - R = -R(1+\sqrt{2}/2).$ Correct. 
$Z = (-R(\sqrt{2}+1), 0).$ On $L_3$: $0 = (\sqrt{2}-1)(-R(\sqrt{2}+1)) + R = -R(2-1) + R = -R+R = 0.$ Correct. On $L_6$: $0 = -(\sqrt{2}-1)(-R(\sqrt{2}+1)) - R = R(2-1) - R = R-R = 0.$ Correct.

Slope of $XZ$: $m_{XZ} = \frac{R(\sqrt{2}+1)/2 - 0}{R/2 - (-R(\sqrt{2}+1))} = \frac{R(\sqrt{2}+1)/2}{R(1/2+\sqrt{2}+1)} = \frac{(\sqrt{2}+1)/2}{3/2+\sqrt{2}} = \frac{\sqrt{2}+1}{3+2\sqrt{2}} = \frac{\sqrt{2}+1}{(1+\sqrt{2})^2} = \frac{1}{1+\sqrt{2}} = \sqrt{2}-1.$ This is the slope of $A_3A_4.$ Oh, I made a mistake earlier. $X, Z$ are on $A_3A_4.$ $Y, Z$ are on $A_6A_7.$ $X, Y$ are on $A_1A_2.$ The triangle $XYZ$ must degenerate to a point if the lines are not parallel. Wait, $X$ is on $L_1$ and $L_3.$ $Y$ is on $L_1$ and $L_6.$ $Z$ is on $L_3$ and $L_6.$ Okay, the points are correct. The slopes of the lines $L_1, L_3, L_6$ are $m_1 = -(1+\sqrt{2}),$ $m_3 = \sqrt{2}-1,$ $m_6 = -(\sqrt{2}-1).$ 
Let's calculate the slopes of the sides of $\triangle XYZ.$ 
Slope of $XY$ is $m_1 = -(1+\sqrt{2}).$ This line passes through $X$ and $Y.$ Correct. 
Slope of $XZ$ is $m_3 = \sqrt{2}-1.$ This line passes through $X$ and $Z.$ Correct. 
Slope of $YZ$ is $m_6 = -(\sqrt{2}-1).$ This line passes through $Y$ and $Z.$ Correct.

The slopes of the sides of the triangle are $m_1, m_3, m_6.$ 
Let's check if $XZ \perp YZ.$ The product of slopes $m_3 m_6 = (\sqrt{2}-1)(-(\sqrt{2}-1)) = -(\sqrt{2}-1)^2 = -(2-2\sqrt{2}+1) = -(3-2\sqrt{2}).$ Not -1. So it's not a right triangle with right angle at Z. 
Let's check if $XY \perp XZ.$ $m_1 m_3 = -(1+\sqrt{2})(\sqrt{2}-1) = -( \sqrt{2}+1)(\sqrt{2}-1) = -(2-1) = -1.$ Yes, the angle at $X$ is a right angle. $\angle YXZ = 90^\circ.$ 
Let's check if it's isosceles. Need to compare side lengths. 
$XZ^2 = (x_Z - x_X)^2 + (y_Z - y_X)^2 = (-R(\sqrt{2}+1) - R/2)^2 + (0 - R(\sqrt{2}+1)/2)^2.$ 
$XZ^2 = R^2(-\sqrt{2}-1 - 1/2)^2 + R^2(-\frac{\sqrt{2}+1}{2})^2 = R^2 (-3/2-\sqrt{2})^2 + R^2 \frac{2+2\sqrt{2}+1}{4}.$ 
$XZ^2 = R^2 (9/4 + 3\sqrt{2} + 2) + R^2 \frac{3+2\sqrt{2}}{4} = R^2 (17/4 + 3\sqrt{2}) + R^2 \frac{3+2\sqrt{2}}{4}.$ 
$XZ^2 = R^2 \frac{17+12\sqrt{2} + 3+2\sqrt{2}}{4} = R^2 \frac{20+14\sqrt{2}}{4} = R^2 \frac{10+7\sqrt{2}}{2}.$ 
$XY^2 = (x_Y - x_X)^2 + (y_Y - y_X)^2 = (R(1+\sqrt{2}/2) - R/2)^2 + (-R(1+\sqrt{2}/2) - R(\sqrt{2}+1)/2)^2.$ 
$XY^2 = R^2(1+\sqrt{2}/2 - 1/2)^2 + R^2(-(1+\sqrt{2}/2) - (\sqrt{2}+1)/2)^2.$ 
$XY^2 = R^2(1/2+\sqrt{2}/2)^2 + R^2(-\frac{2+\sqrt{2}}{2} - \frac{\sqrt{2}+1}{2})^2 = R^2(\frac{1+\sqrt{2}}{2})^2 + R^2(-\frac{3+2\sqrt{2}}{2})^2.$ 
$XY^2 = R^2 \frac{1+2\sqrt{2}+2}{4} + R^2 \frac{9+12\sqrt{2}+8}{4} = R^2 \frac{3+2\sqrt{2}}{4} + R^2 \frac{17+12\sqrt{2}}{4}.$ 
$XY^2 = R^2 \frac{20+14\sqrt{2}}{4} = R^2 \frac{10+7\sqrt{2}}{2}.$ 
So $XY^2 = XZ^2.$ Thus $\triangle XYZ$ is an isosceles right triangle with right angle at $X.$ This matches the claim in the provided solution.

The solution continues: "the height with regard to which shall be $3+2\sqrt{2}$." Which height? Height from $X$ to $YZ$? $X$ is the right angle vertex. Let's check the distance from $X$ to the line $YZ$ (which has equation $y = -(\sqrt{2}-1)x - R$ ). Line is $(\sqrt{2}-1)x + y + R = 0.$ 
Point $X=(R/2, R(\sqrt{2}+1)/2).$ 
Distance $h = \frac{|(\sqrt{2}-1)R/2 + R(\sqrt{2}+1)/2 + R|}{\sqrt{(\sqrt{2}-1)^2 + 1^2}} = \frac{|R(\sqrt{2}-1 + \sqrt{2}+1)/2 + R|}{\sqrt{2-2\sqrt{2}+1 + 1}} = \frac{|R(2\sqrt{2}/2) + R|}{\sqrt{4-2\sqrt{2}}} = \frac{|R\sqrt{2} + R|}{\sqrt{4-2\sqrt{2}}} = \frac{R(\sqrt{2}+1)}{\sqrt{4-2\sqrt{2}}}.$ 
This doesn't look like $3+2\sqrt{2}.$ Maybe the height is the length of the leg $XY$ or $XZ$? 
$XY = \sqrt{R^2 \frac{10+7\sqrt{2}}{2}}.$ Recall $R^2 = 4+2\sqrt{2}.$ 
$XY^2 = (4+2\sqrt{2}) \frac{10+7\sqrt{2}}{2} = (2+\sqrt{2})(10+7\sqrt{2}) = 20 + 14\sqrt{2} + 10\sqrt{2} + 14 = 34 + 24\sqrt{2}.$ 
$XY = \sqrt{34+24\sqrt{2}} = \sqrt{34+ \sqrt{2 \times 144 \times 4}} = \sqrt{34+\sqrt{1152}}.$ Not simple. 
Let's try simplifying $34+24\sqrt{2} = 34+\sqrt{576 \times 2} = 34+\sqrt{1152}.$ Can we write this in the form $(a+b\sqrt{2})^2 = a^2+2b^2 + 2ab\sqrt{2}$? 
$2ab = 24 \implies ab=12.$ Possible integer pairs $(1,12), (2,6), (3,4).$ 
If $(a,b)=(3,4),$ $a^2+2b^2 = 3^2 + 2(4^2) = 9 + 2(16) = 9+32=41.$ No. 
If $(a,b)=(4,3),$ $a^2+2b^2 = 4^2 + 2(3^2) = 16 + 2(9) = 16+18=34.$ Yes. 
So $XY = 4+3\sqrt{2}.$ 
What is $3+2\sqrt{2}$? Let's see. The distance $R$ was calculated using side length $s=2.$ $R^2 = 4+2\sqrt{2}.$ 
$3+2\sqrt{2} = (\sqrt{2}+1)^2.$ Is this related to $XY?$ $XY=4+3\sqrt{2}.$ No relation.

Let's rethink the height part. Maybe it's the height of the point $P$ relative to the triangle $XYZ.$ 
The solution mentions $PU, PV, PW.$ $PU$ is the distance from $P$ to the line $A_1A_2.$ $PV$ is the distance from $P$ to $A_3A_4.$ $PW$ is the distance from $P$ to $A_6A_7.$ 
Let $P=(x_p, y_p).$ The lines $L_1, L_3, L_6$ form the triangle $XYZ.$ 
$L_1: (1+\sqrt{2})x + y - R(1+\sqrt{2}) = 0.$ 
$L_3: (\sqrt{2}-1)x - y + R = 0.$ 
$L_6: (\sqrt{2}-1)x + y + R = 0.$ 
Let the coordinates of $P$ be $(x_p, y_p).$ 
$PU = A(T_1) = \frac{1}{2} s h_1.$ Since $s=2,$ $PU=h_1.$ It's the height of $\triangle PA_1A_2,$ not the perpendicular distance from $P$ to the line $L_1.$ Area $A(T_1) = \frac{1}{2} \times \text{base} \times \text{height}.$ $A(T_1) = \frac{1}{2} \times 2 \times PU = PU.$ Ok, so $PU$ is the height. Height from P to the segment $A_1A_2.$

The equation $\frac{PU}{\sqrt{2}} + \frac{PV}{\sqrt{2}} + PW = 3+2\sqrt2$ is stated without proof. Let's try to understand it. 
This looks like a coordinate system or some geometric property. 
The lines $L_1, L_3, L_6$ have slopes $-(1+\sqrt{2}), \sqrt{2}-1, -(\sqrt{2}-1).$ 
$L_3$ has angle $\arctan(\sqrt{2}-1).$ $\tan(\pi/8) = \sqrt{2}-1.$ So $L_3$ has angle $\pi/8$ with the x-axis. 
$L_6$ has angle $\arctan(-(\sqrt{2}-1)) = -\pi/8.$ 
$L_1$ has angle $\arctan(-(1+\sqrt{2})).$ $\tan(3\pi/8) = \sqrt{2}+1.$ So $L_1$ has angle $-\arctan(1+\sqrt{2}) = -3\pi/8.$ Wait, the slope is $-(1+\sqrt{2}).$ $\tan \alpha = -(1+\sqrt{2}).$ $\alpha = \pi - 3\pi/8 = 5\pi/8$? No, $\alpha$ is in $(-\pi/2, \pi/2).$ For $L_1$: $y = -(1+\sqrt{2})x + R(1+\sqrt{2}).$ Angle is $\pi - 3\pi/8.$ No, the angle is $3\pi/8 + \pi/2 = 7\pi/8$? Let's check the angle $A_1A_2.$ $A_1=(R,0), A_2=(R\cos(\pi/4), R\sin(\pi/4)).$ The segment $A_1A_2$ has angle $3\pi/8$ with the radius $OA_1.$ The angle of $OA_1$ is 0. Angle of $OA_2$ is $\pi/4.$ The angle of the chord $A_1A_2$ is $(\theta_1+\theta_2)/2 = \pi/8.$ The slope is $\tan(\pi/8)?$ No, that's the angle of the radius bisecting the segment. The angle of the segment itself relative to the x-axis. The midpoint is $( (R+R\sqrt{2}/2)/2, (R\sqrt{2}/2)/2 ).$ Slope is $\frac{R\sqrt{2}/2}{R\sqrt{2}/2 - R} = \frac{\sqrt{2}/2}{\sqrt{2}/2-1} = \frac{\sqrt{2}}{\sqrt{2}-2} = \frac{\sqrt{2}(\sqrt{2}+2)}{2-4} = \frac{2+2\sqrt{2}}{-2} = -(1+\sqrt{2}).$ This matches $m_1.$ The angle is $\alpha_1.$ $\tan \alpha_1 = -(1+\sqrt{2}).$ $\alpha_1 = \pi - 3\pi/8 = 5\pi/8.$ Or maybe $-3\pi/8.$ 
$L_3$: $A_3=(0,R), A_4=(-R\sqrt{2}/2, R\sqrt{2}/2).$ Slope is $\frac{R\sqrt{2}/2-R}{-R\sqrt{2}/2} = \frac{\sqrt{2}/2-1}{-\sqrt{2}/2} = \frac{\sqrt{2}-2}{-\sqrt{2}} = \frac{2-\sqrt{2}}{\sqrt{2}} = \sqrt{2}-1.$ This matches $m_3.$ The angle is $\alpha_3 = \pi/8.$ 
$L_6$: $A_6=(-R\sqrt{2}/2, -R\sqrt{2}/2), A_7=(0,-R).$ Slope is $\frac{-R - (-R\sqrt{2}/2)}{0 - (-R\sqrt{2}/2)} = \frac{-R(1-\sqrt{2}/2)}{R\sqrt{2}/2} = \frac{-(2-\sqrt{2})/2}{\sqrt{2}/2} = \frac{-(2-\sqrt{2})}{\sqrt{2}} = \frac{\sqrt{2}-2}{\sqrt{2}} = 1-\sqrt{2}.$ This is $-m_3 = -(\sqrt{2}-1).$ Matches $m_6.$ The angle is $\alpha_6 = -\pi/8.$

Let's consider the coordinate system defined by the lines $L_3$ and $L_6.$ They intersect at $Z=(-R(\sqrt{2}+1), 0).$ The lines are $L_3: y = (\sqrt{2}-1)(x+R(\sqrt{2}+1)).$ $L_6: y = -(\sqrt{2}-1)(x+R(\sqrt{2}+1)).$ 
The angle between $L_3$ and $L_6$ is $\pi/8 - (-\pi/8) = \pi/4.$ 
The line $L_1$ intersects these two lines. The angle between $L_1$ and $L_3$ is $\pi/8 - (5\pi/8) = -4\pi/8 = -\pi/2.$ Yes, they are perpendicular, angle at $X.$ 
The angle between $L_1$ and $L_6$ is $-\pi/8 - (5\pi/8) = -6\pi/8 = -3\pi/4.$ This angle inside the triangle is $\pi-|-3\pi/4|=\pi/4.$ 
The angle between $L_3$ and $L_6$ at $Z$ is $\pi/4.$ 
The angles of $\triangle XYZ$ are $X:\pi/2,$ $Y:\pi/4,$ $Z:\pi/4.$ It's an isosceles right triangle. Okay, this matches.

The equation $\frac{PU}{\sqrt{2}} + \frac{PV}{\sqrt{2}} + PW = 3+2\sqrt2$ seems to relate distances from $P$ to the sides. But $PU, PV, PW$ are heights of the triangles $\triangle PA_1A_2,$ etc. not distances to the lines $L_1, L_3, L_6.$ 
Is it possible that $P$ is constrained to be on some line or circle? $P$ is just a point inside the circle. 
Let the signed distances from $P$ to $L_1, L_3, L_6$ be $d_1, d_3, d_6.$ 
Let's assume the formula uses the perpendicular distances from $P$ to the lines $L_1, L_3, L_6.$ Call them $p_1, p_3, p_6.$ 
Is there a known formula relating distances from a point to the sides of a triangle? 
For an equilateral triangle, the sum of distances from any interior point to the sides is constant (equal to the altitude). Not equilateral here.

Let's check the quantity $3+2\sqrt{2}.$ $R^2 = 4+2\sqrt{2}.$ Side length $s=2.$ Altitude of $\triangle OA_1A_2$ from $O$ to $A_1A_2$ is $h = R \cos(\pi/8).$ $\cos^2(\pi/8) = (1+\cos(\pi/4))/2 = (1+\sqrt{2}/2)/2 = (2+\sqrt{2})/4.$ $\cos(\pi/8) = \sqrt{2+\sqrt{2}}/2.$ 
$h = \sqrt{4+2\sqrt{2}} \frac{\sqrt{2+\sqrt{2}}}{2} = \sqrt{2(2+\sqrt{2})} \frac{\sqrt{2+\sqrt{2}}}{2} = \frac{\sqrt{2}\sqrt{2+\sqrt{2}}\sqrt{2+\sqrt{2}}}{2} = \frac{\sqrt{2}(2+\sqrt{2})}{2} = \frac{2\sqrt{2}+2}{2} = \sqrt{2}+1.$ This is the distance from O to the side $A_1A_2.$ 
Let $M_i$ be the midpoint of $A_iA_{i+1}.$ $OM_i = \sqrt{2}+1.$ 
The line $L_1$ contains segment $A_1A_2.$ Its distance from $O(0,0)$ is $d(O, L_1) = \frac{|-(1+\sqrt{2})0 + 0 - R(1+\sqrt{2})|}{\sqrt{(1+\sqrt{2})^2+1^2}} = \frac{R(1+\sqrt{2})}{\sqrt{1+2\sqrt{2}+2+1}} = \frac{R(1+\sqrt{2})}{\sqrt{4+2\sqrt{2}}}.$ 
$d(O, L_1) = \frac{R(1+\sqrt{2})}{\sqrt{R^2}} = R(1+\sqrt{2})/R = 1+\sqrt{2}.$ This works. The distance from $O$ to the line containing $A_iA_{i+1}$ is $h=\sqrt{2}+1.$

Let's assume $P=(0,0)=O.$ Then $A(S_i) = \text{Area(Sector } OA_iA_{i+1}) = 1/8.$ 
$A(S_1)=1/8,$ but given $1/7.$ $A(S_3)=1/8,$ but given $1/9.$ So $P \neq O.$ 
In my formula $A(S_i) = 1/8 - C \cos(\alpha_i - \phi).$ 
$A(S_1)=1/7 \implies C \cos(\alpha_1 - \phi) = -1/56.$ 
$A(S_3)=1/9 \implies C \cos(\alpha_3 - \phi) = 1/72.$ $\alpha_3 = \alpha_1 + \pi/2.$ $C \cos(\alpha_1 - \phi + \pi/2) = -C \sin(\alpha_1 - \phi) = 1/72.$ So $C \sin(\alpha_1 - \phi) = -1/72.$ 
$A(S_6)=1/8 - C \cos(\alpha_6 - \phi).$ $\alpha_6 = \alpha_1 + 5\pi/4.$ $C \cos(\alpha_1 - \phi + 5\pi/4) = C (\cos(\alpha_1-\phi)\cos(5\pi/4) - \sin(\alpha_1-\phi)\sin(5\pi/4)).$ 
$C \cos(\psi+5\pi/4) = C (\cos\psi (-\sqrt{2}/2) - \sin\psi (-\sqrt{2}/2)) = (-\sqrt{2}/2) (C \cos\psi + C \sin\psi).$ 
$= (-\sqrt{2}/2) (-1/56 - 1/72) = (-\sqrt{2}/2) (- (9+7)/504) = (-\sqrt{2}/2) (-16/504) = (\sqrt{2}/2) (16/504) = (\sqrt{2}/2)(2/63) = \sqrt{2}/63.$ 
$A(S_6) = 1/8 - \sqrt{2}/63.$ This was my result. It means $n=63.$

Let's re-read the provided solution carefully. 
$PU = (\frac{1}{7}-\frac{1}{8}) \pi (4+ 2\sqrt{2}) + \sqrt{2}+1.$ This is area $A(T_1).$ 
$PV = (\frac{1}{9}-\frac{1}{8}) \pi (4+ 2\sqrt{2}) + \sqrt{2}+1.$ This is area $A(T_3).$ 
$PW$ is presumably the area $A(T_6).$ $A(T_6) = A(S_6) - D = A(S_6) - (\frac{\pi R^2}{8} - (\sqrt{2}+1)).$ 
The formula $\frac{PU}{\sqrt{2}} + \frac{PV}{\sqrt{2}} + PW = 3+2\sqrt2$ is critical and unproven. Let's assume it's correct and see where it leads. 
In the formula, $PU, PV, PW$ are areas $A(T_1), A(T_3), A(T_6).$ 
Let $A_{circ} = \pi R^2 = \pi(4+2\sqrt{2}).$ 
$A(T_1) = \frac{1}{56} A_{circ} + (\sqrt{2}+1).$ 
$A(T_3) = -\frac{1}{72} A_{circ} + (\sqrt{2}+1).$ 
$A(T_6)$ is what we need to find $A(S_6).$ Let $A(T_6)=x.$ 
$\frac{1}{\sqrt{2}} (\frac{1}{56} A_{circ} + \sqrt{2}+1) + \frac{1}{\sqrt{2}} (-\frac{1}{72} A_{circ} + \sqrt{2}+1) + x = 3+2\sqrt{2}.$ 
$\frac{A_{circ}}{\sqrt{2}} (\frac{1}{56} - \frac{1}{72}) + \frac{1}{\sqrt{2}}(\sqrt{2}+1 + \sqrt{2}+1) + x = 3+2\sqrt{2}.$ 
$\frac{A_{circ}}{\sqrt{2}} (\frac{72-56}{56 \times 72}) + \frac{2\sqrt{2}+2}{\sqrt{2}} + x = 3+2\sqrt{2}.$ 
$\frac{A_{circ}}{\sqrt{2}} (\frac{16}{4032}) + (2 + \sqrt{2}) + x = 3+2\sqrt{2}.$ 
$16/4032 = 1/252.$ 
$\frac{A_{circ}}{252\sqrt{2}} + 2 + \sqrt{2} + x = 3+2\sqrt{2}.$ 
$x = 3+2\sqrt{2} - (2+\sqrt{2}) - \frac{A_{circ}}{252\sqrt{2}}.$ 
$x = 1+\sqrt{2} - \frac{\pi(4+2\sqrt{2})}{252\sqrt{2}}.$ 
$x = 1+\sqrt{2} - \frac{\pi(4\sqrt{2}+4)}{504}.$ 
$A(T_6) = 1+\sqrt{2} - \frac{\pi(4\sqrt{2}+4)}{504}.$ 
This is the area of triangle $\triangle PA_6A_7.$ 
Now, we need $A(S_6) = A(T_6) + D.$ 
$A(S_6) = 1+\sqrt{2} - \frac{\pi(4\sqrt{2}+4)}{504} + \frac{\pi R^2}{8} - (\sqrt{2}+1).$ 
$A(S_6) = 1+\sqrt{2} - \frac{\pi(4\sqrt{2}+4)}{504} + \frac{\pi (4+2\sqrt{2})}{8} - \sqrt{2}-1.$ 
$A(S_6) = \frac{\pi(4+2\sqrt{2})}{8} - \frac{\pi(4\sqrt{2}+4)}{504}.$ 
$A(S_6) = \frac{\pi(2+\sqrt{2})}{4} - \frac{\pi(4\sqrt{2}+4)}{504}.$ 
$A(S_6) = \pi [ \frac{2+\sqrt{2}}{4} - \frac{4\sqrt{2}+4}{504} ].$ 
$A(S_6) = \pi [ \frac{126(2+\sqrt{2})}{504} - \frac{4\sqrt{2}+4}{504} ].$ 
$A(S_6) = \pi [ \frac{252 + 126\sqrt{2} - 4\sqrt{2} - 4}{504} ].$ 
$A(S_6) = \pi [ \frac{248 + 122\sqrt{2}}{504} ].$ 
$A(S_6) = \frac{\pi (124 + 61\sqrt{2})}{252}.$

The result is required in the form $\frac{A_{circ}}{8} - \frac{\sqrt{2}}{n} A_{circ}.$ No, the form is $1/8 - \sqrt{2}/n,$ where the total area is $1.$ 
Let's rescale the result by dividing by $A_{circ} = \pi(4+2\sqrt{2}).$ 
$\frac{A(S_6)}{A_{circ}} = \frac{\pi (124 + 61\sqrt{2})/252}{\pi(4+2\sqrt{2})} = \frac{124+61\sqrt{2}}{252(4+2\sqrt{2})} = \frac{124+61\sqrt{2}}{252 \times 2 (2+\sqrt{2})} = \frac{124+61\sqrt{2}}{504(2+\sqrt{2})}.$ 
$\frac{124+61\sqrt{2}}{504(2+\sqrt{2})} = \frac{(124+61\sqrt{2})(2-\sqrt{2})}{504(4-2)} = \frac{248 - 124\sqrt{2} + 122\sqrt{2} - 122}{504 \times 2}.$ 
$= \frac{126 - 2\sqrt{2}}{1008} = \frac{63 - \sqrt{2}}{504}.$ 
Is this $\frac{1}{8} - \frac{\sqrt{2}}{n}$? 
$\frac{1}{8} - \frac{\sqrt{2}}{n} = \frac{63}{504} - \frac{\sqrt{2}}{n}.$ 
Comparing $\frac{63 - \sqrt{2}}{504} = \frac{63}{504} - \frac{\sqrt{2}}{504}.$ 
This implies $n=504.$

So the alternative solution gives $n=504.$ My solution gives $n=63.$ One of them must be wrong. Let me check the derivation again. My derivation $A(S_i) = 1/8 - C \cos(\alpha_i - \phi)$ seems correct. The calculation steps also seem correct. 
Let's re-check the calculation: 
$C \cos \psi = -1/56.$ $C \sin \psi = -1/72.$ 
$C \cos(\psi + 5\pi/4) = (-\sqrt{2}/2) (C \cos \psi + C \sin \psi)$ 
$= (-\sqrt{2}/2) (-1/56 - 1/72) = (-\sqrt{2}/2) (- (1/56+1/72))$ 
$1/56 + 1/72 = \frac{9+7}{504} = 16/504 = 2/63.$ 
$= (-\sqrt{2}/2) (-2/63) = \sqrt{2}/63.$ 
$A(S_6) = 1/8 - C \cos(\psi + 5\pi/4) = 1/8 - \sqrt{2}/63.$ So $n=63.$

Let's re-check the alternative solution calculation. 
$PW = 1+\sqrt{2}- \frac{1}{\sqrt{2}}\left(\frac{1}{7}+\frac{1}{9}-\frac{1}{4}\right)\pi\left(4+2\sqrt{2}\right)$ 
$= 1+\sqrt{2} - \frac{1}{\sqrt{2}} (\frac{9+7}{63} - \frac{1}{4}) \pi (4+2\sqrt{2})$ 
$= 1+\sqrt{2} - \frac{1}{\sqrt{2}} (\frac{16}{63} - \frac{1}{4}) \pi (4+2\sqrt{2}).$ 
Okay, calculation in the solution provided: 
$PW = 3+2\sqrt2-\frac{1}{\sqrt{2}}(PU+PV)$ 
$PU+PV = (\frac{1}{56} - \frac{1}{72}) A_{circ} + 2(\sqrt{2}+1).$ 
$PW = 3+2\sqrt{2} - \frac{1}{\sqrt{2}} [ (\frac{1}{56} - \frac{1}{72}) A_{circ} + 2(\sqrt{2}+1) ].$ 
$PW = 3+2\sqrt{2} - \frac{A_{circ}}{\sqrt{2}} (\frac{16}{4032}) - \frac{2(\sqrt{2}+1)}{\sqrt{2}}.$ 
$PW = 3+2\sqrt{2} - \frac{A_{circ}}{252\sqrt{2}} - (2+\sqrt{2}).$ 
$PW = 1+\sqrt{2} - \frac{A_{circ}}{252\sqrt{2}}.$ This matches my calculation of $x=A(T_6)$ above. 
$A(T_6) = 1+\sqrt{2} - \frac{\pi(4+2\sqrt{2})}{252\sqrt{2}} = 1+\sqrt{2} - \frac{\pi(4\sqrt{2}+4)}{504}.$ 
Then $A(S_6) = A(T_6) + D = 1+\sqrt{2} - \frac{\pi(4\sqrt{2}+4)}{504} + \frac{A_{circ}}{8} - (\sqrt{2}+1).$ 
$A(S_6) = \frac{A_{circ}}{8} - \frac{\pi(4\sqrt{2}+4)}{504}.$ 
$A(S_6) = \frac{\pi(4+2\sqrt{2})}{8} - \frac{\pi(4\sqrt{2}+4)}{504}.$ 
Let's scale this to $A_{circ}=1.$ $\pi R^2=1.$ 
Rescaled $A(S_6) = 1/8 - \frac{\pi(4\sqrt{2}+4)}{504} / (\pi R^2) = 1/8 - \frac{\pi(4\sqrt{2}+4)}{504} / (\pi(4+2\sqrt{2})).$ 
$A(S_6) = 1/8 - \frac{4(\sqrt{2}+1)}{504(2)(2+\sqrt{2})} = 1/8 - \frac{2(\sqrt{2}+1)}{504(2+\sqrt{2})}.$ 
$A(S_6) = 1/8 - \frac{\sqrt{2}(\sqrt{2}+1)}{252(2+\sqrt{2})} = 1/8 - \frac{2+\sqrt{2}}{252(2+\sqrt{2})} = 1/8 - \frac{1}{252}.$ 
This is not in the form $1/8 - \sqrt{2}/n.$

Let's check the calculation in the provided solution text itself. 
$\text{Target Area} = \frac{1}{8} \pi \left(4+2\sqrt{2}\right)- \left(\sqrt{2}+1\right) + \left(1+\sqrt{2}\right)- \frac{1}{\sqrt{2}}\left(\frac{1}{7}+\frac{1}{9}-\frac{1}{4}\right)\pi\left(4+2\sqrt{2}\right).$ 
This step computes $D + PW.$ $A(T_6)$ is denoted as $PW$ in the text, which is area( $\triangle PA_6A_7$ ). 
Area $= D + A(T_6).$ 
Area $= \frac{A_{circ}}{8} - (\sqrt{2}+1) + PW.$ 
$PW = 1+\sqrt{2}- \frac{1}{\sqrt{2}}\left(\frac{1}{7}+\frac{1}{9}-\frac{1}{4}\right) A_{circ}.$ This parenthesis term is calculation error in the provided text. 
Let me recalculate PW from the equation $\frac{PU}{\sqrt{2}} + \frac{PV}{\sqrt{2}} + PW = 3+2\sqrt2.$ 
$PU = A(T_1) = (\frac{1}{7}-\frac{1}{8})A_{circ} + \sqrt{2}+1.$ 
$PV = A(T_3) = (\frac{1}{9}-\frac{1}{8})A_{circ} + \sqrt{2}+1.$ 
$PW = A(T_6).$ 
$\frac{1}{\sqrt{2}} [ (\frac{1}{56}) A_{circ} + \sqrt{2}+1 ] + \frac{1}{\sqrt{2}} [ (-\frac{1}{72}) A_{circ} + \sqrt{2}+1 ] + PW = 3+2\sqrt{2}.$ 
$\frac{A_{circ}}{\sqrt{2}} (\frac{1}{56}-\frac{1}{72}) + \frac{1}{\sqrt{2}} (2\sqrt{2}+2) + PW = 3+2\sqrt{2}.$ 
$\frac{A_{circ}}{\sqrt{2}} (\frac{16}{4032}) + (2+\sqrt{2}) + PW = 3+2\sqrt{2}.$ 
$\frac{A_{circ}}{252\sqrt{2}} + 2+\sqrt{2} + PW = 3+2\sqrt{2}.$ 
$PW = 1+\sqrt{2} - \frac{A_{circ}}{252\sqrt{2}}.$ 
$A(S_6) = D + PW = (\frac{A_{circ}}{8} - (\sqrt{2}+1)) + (1+\sqrt{2} - \frac{A_{circ}}{252\sqrt{2}}).$ 
$A(S_6) = \frac{A_{circ}}{8} - \frac{A_{circ}}{252\sqrt{2}}.$ 
Rescale to $A_{circ}=1.$ 
$A(S_6) = 1/8 - \frac{1}{252\sqrt{2}} = 1/8 - \frac{\sqrt{2}}{504}.$ 
This leads to $n=504.$

Where is the difference between my initial approach and this? 
My approach: $A(S_i) = 1/8 + \mathcal{A}(OPA_i) + \mathcal{A}(OA_{i+1}P) = 1/8 + \frac{1}{2} (\vec{p} \times \vec{a_i} - \vec{p} \times \vec{a_{i+1}}).$ 
The alternative approach: $A(S_i) = A(T_i) + D.$ Where $A(T_i) = \text{Area}(\triangle PA_iA_{i+1})$ and $D = \text{Area(Segment)}.$ 
$A(T_i) = A(S_i) - D.$ 
$A(T_i) = A(S_i) - (\text{Area(Sector)} - \text{Area}(\triangle OA_iA_{i+1})).$ 
Let $A_{sec} = 1/8.$ Let $A_{triO} = \text{Area}(\triangle OA_iA_{i+1}).$ 
$A(T_i) = A(S_i) - A_{sec} + A_{triO}.$

The formula $\frac{PU}{\sqrt{2}} + \frac{PV}{\sqrt{2}} + PW = 3+2\sqrt2$ is stated for $PU=A(T_1), PV=A(T_3), PW=A(T_6).$ 
$\frac{A(T_1)}{\sqrt{2}} + \frac{A(T_3)}{\sqrt{2}} + A(T_6) = 3+2\sqrt{2}.$ Let's check the units. $A(T)$ is area. $3+2\sqrt{2}$ is... $XY = 4+3\sqrt{2}.$ $(1+\sqrt{2})^2 = 3+2\sqrt{2}.$

Let's reconsider the formula $\frac{PU}{\sqrt{2}} + \frac{PV}{\sqrt{2}} + PW = 3+2\sqrt2.$ Maybe $PU, PV, PW$ are indeed heights, not areas. $A(T_i) = \frac{1}{2} s h_i.$ With $s=2,$ $A(T_i)=h_i.$ So $PU, PV, PW$ are heights. 
Let $h_1, h_3, h_6$ be the heights of triangles $\triangle PA_1A_2, \triangle PA_3A_4, \triangle PA_6A_7$ from P to the respective sides. 
$h_1 = A(T_1) = A(S_1) - D = A_{circ}/7 - D.$ 
$h_3 = A(T_3) = A(S_3) - D = A_{circ}/9 - D.$ 
$h_6 = A(T_6) = A(S_6) - D.$ 
$A_{circ} = \pi(4+2\sqrt{2}), D = \frac{A_{circ}}{8} - (\sqrt{2}+1).$ 
$h_1 = (\frac{1}{7}-\frac{1}{8}) A_{circ} + \sqrt{2}+1.$ 
$h_3 = (\frac{1}{9}-\frac{1}{8}) A_{circ} + \sqrt{2}+1.$ 
$h_6 = A(S_6) - \frac{A_{circ}}{8} + \sqrt{2}+1.$ 
Substitute these into the formula $\frac{h_1}{\sqrt{2}} + \frac{h_3}{\sqrt{2}} + h_6 = 3+2\sqrt{2}.$ 
$\frac{1}{\sqrt{2}} [ (\frac{1}{56}) A_{circ} + \sqrt{2}+1 ] + \frac{1}{\sqrt{2}} [ (-\frac{1}{72}) A_{circ} + \sqrt{2}+1 ] + h_6 = 3+2\sqrt{2}.$ 
$\frac{A_{circ}}{252\sqrt{2}} + \frac{2\sqrt{2}+2}{\sqrt{2}} + h_6 = 3+2\sqrt{2}.$ 
$\frac{A_{circ}}{252\sqrt{2}} + 2+\sqrt{2} + h_6 = 3+2\sqrt{2}.$ 
$h_6 = 1+\sqrt{2} - \frac{A_{circ}}{252\sqrt{2}}.$ 
Now, substitute $h_6 = A(S_6) - A_{circ}/8 + \sqrt{2}+1.$ 
$A(S_6) - A_{circ}/8 + \sqrt{2}+1 = 1+\sqrt{2} - \frac{A_{circ}}{252\sqrt{2}}.$ 
$A(S_6) = A_{circ}/8 - \frac{A_{circ}}{252\sqrt{2}}.$ 
This is the same result as before. The calculation seems to be consistent within the alternative solution's framework.

Where does the formula $\frac{h_1}{\sqrt{2}} + \frac{h_3}{\sqrt{2}} + h_6 = 3+2\sqrt{2}$ come from? 
Is it related to the geometry of the points $A_1, A_2, A_3, A_4, A_6, A_7$? 
Let's set up a coordinate system. $O=(0,0).$ Let $P=(x,y).$ 
$A_k = (R \cos \theta_k, R \sin \theta_k).$ 
The line $L_k$ containing $A_k A_{k+1}.$ The height $h_k$ is the distance from $P$ to the segment $A_k A_{k+1}.$ 
Area( $\triangle PA_kA_{k+1}) = \frac{1}{2} |\vec{PA_k} \times \vec{PA_{k+1}}|.$ 
The height $h_k = \frac{2 \text{Area}(\triangle PA_kA_{k+1})}{|\vec{A_kA_{k+1}}|} = \frac{2 A(T_k)}{s}.$ Since $s=2,$ $h_k=A(T_k).$ 
Okay, so the relation really is for areas $A(T_k).$

Let's re-examine my model $A(S_i) = 1/8 - C \cos(\alpha_i - \phi).$ Is it possible this formula is wrong? 
Area( $S_i$ ) is the area of the region bounded by $PA_i$, $PA_{i+1}$ and arc $A_i A_{i+1}$. 
My calculation is based on Green's theorem: $\oint x dy - y dx.$ I integrated along $P \to A_i \to A_{i+1} \to P.$

\begin{enumerate}\item $P \to A_i$: $\frac{1}{2}(x_p y_i - y_p x_i) = \mathcal{A}(OPA_i).$\item Arc $A_i \to A_{i+1}$: $\text{Area(Sector } OA_iA_{i+1}) = 1/8.$\item $A_{i+1} \to P$: $\frac{1}{2}(x_{i+1} y_p - y_{i+1} x_p) = \mathcal{A}(OA_{i+1}P).$ 
The sum is $A(S_i) = 1/8 + \mathcal{A}(OPA_i) + \mathcal{A}(OA_{i+1}P).$ 
Area calculation seems correct. 
$\mathcal{A}(OPA_i) = \frac{1}{2} |\vec{p}| |\vec{a_i}| \sin(\angle(\vec{p}, \vec{a_i})) = \frac{1}{2} r R \sin(\theta_i - \phi).$ 
$\mathcal{A}(OA_{i+1}P) = \frac{1}{2} |\vec{a_{i+1}}| |\vec{p}| \sin(\angle(\vec{a_{i+1}}, \vec{p})) = \frac{1}{2} R r \sin(\phi - \theta_{i+1}) = -\frac{1}{2} rR \sin(\theta_{i+1}-\phi).$ 
$A(S_i) = 1/8 + \frac{1}{2} rR (\sin(\theta_i-\phi) - \sin(\theta_{i+1}-\phi)).$ 
This expression matches what I derived before. 
$A(S_i) = 1/8 - rR \sin(\pi/8) \cos(\frac{\theta_i+\theta_{i+1}}{2} - \phi).$ 
This formula is correct. 
All the subsequent calculations based on this formula lead to $n=63.$
\end{enumerate}

Let's rethink the area interpretation. 
Area( $S_i$ ) = Area( $\triangle PA_iA_{i+1}$ ) + Area(Segment $A_iA_{i+1}$ ). 
Area( $S_i$ ) = Area( $\triangle PA_iA_{i+1}$ ) + $D.$ 
Area( $\triangle PA_iA_{i+1}$ ) = Area( $S_i$ ) - $D.$ 
Area( $\triangle PA_iA_{i+1}$ ) = $A(S_i) - (\text{Area(Sector)} - \text{Area}(\triangle OA_iA_{i+1})).$ 
$A(T_i) = A(S_i) - (1/8 - A_{triO}).$ 
$A(T_1) = 1/7 - (1/8 - A_{triO}) = 1/56 + A_{triO}.$ 
$A(T_3) = 1/9 - (1/8 - A_{triO}) = -1/72 + A_{triO}.$ 
$A(T_6) = A(S_6) - (1/8 - A_{triO}).$

Let's assume the side length $s=A_iA_{i+1}$ is used. Let $h_i$ be the height of $\triangle PA_iA_{i+1}$ from $P.$ 
$A(T_i) = \frac{1}{2} s h_i.$ 
The alternative solution sets $s=2.$ Then $A(T_i)=h_i.$ 
It uses the relation $\frac{h_1}{\sqrt{2}} + \frac{h_3}{\sqrt{2}} + h_6 = 3+2\sqrt{2}.$ 
Let's use the scaled areas, where total area is 1. $A_{circ}=1.$ 
$A_{triO} = \text{Area}(\triangle OA_1A_2).$ In scaled units, $A_{triO} = \frac{\text{Area}(\triangle OA_1A_2)}{A_{circ}}.$ 
Area( $\triangle OA_1A_2$ ) = $\sqrt{2}+1.$ Area(circle) = $\pi(4+2\sqrt{2}).$ 
$A_{triO} = \frac{\sqrt{2}+1}{\pi(4+2\sqrt{2})}.$ This is messy.

Let's reconsider the coordinate idea in the alternative solution. 
Let $P$ have coordinates $(x,y).$ The height $h_k$ of $P$ to $A_k A_{k+1}$ could be related to the coordinates. 
The line $L_k$ containing $A_k A_{k+1}.$ Let its equation be $a_k x + b_k y + c_k = 0.$ Then the perpendicular distance from $P(x,y)$ to $L_k$ is $d_k = \frac{|a_k x + b_k y + c_k|}{\sqrt{a_k^2+b_k^2}}.$ 
The height $h_k$ is related to $d_k$ but not identical. $h_k$ is the height relative to the segment $A_k A_{k+1}.$ If the projection of $P$ onto the line $L_k$ falls outside the segment $A_k A_{k+1},$ then $h_k$ is not the perpendicular distance $d_k.$ 
Area $A(T_k) = \frac{1}{2} s h_k.$

The formula $\frac{h_1}{\sqrt{2}} + \frac{h_3}{\sqrt{2}} + h_6 = 3+2\sqrt{2}$ might be specific to the geometry of the octagon sides. $L_1, L_3, L_6$ form an isosceles right triangle $XYZ.$ 
What is $3+2\sqrt{2}$? $3+2\sqrt{2} = (1+\sqrt{2})^2.$ What is $1+\sqrt{2}$? It's the distance from $O$ to the lines $L_k.$ $h=1+\sqrt{2}.$ So $3+2\sqrt{2}=h^2.$ 
Perhaps the relation is about distances $d_k?$ 
Let's assume $P$ is inside the triangle $XYZ.$ Let $P=(x_p, y_p).$ The coordinates $X, Y, Z$ are known. Let's use a coordinate system where $X=(0,0).$ The line $XY$ is $L_1,$ along the y-axis? $L_1$ has slope $-(1+\sqrt{2}).$ $L_3$ has slope $\sqrt{2}-1.$ Since $m_1 m_3 = -1,$ they are perpendicular. Let $L_3$ be the x-axis. $L_1$ be the y-axis. 
Equation of $L_3$: $y=0.$ Equation of $L_1$: $x=0.$ Then $X=(0,0).$ 
This is a rotated coordinate system. Let's call coordinates $(x', y').$ 
$P=(x'_p, y'_p).$ Perpendicular distance to $L_3$ is $|y'_p|.$ Perpendicular distance to $L_1$ is $|x'_p|.$ 
The line $L_6$ forms the hypotenuse $YZ.$ What is its equation in this system? 
The original slopes are $m_1=-(1+\sqrt{2}), m_3=\sqrt{2}-1, m_6=-(\sqrt{2}-1).$ 
Angle of $L_3$ is $\theta_3 = \pi/8.$ Angle of $L_1$ is $\theta_1 = 5\pi/8$? No, $\theta_1$ is the angle for $y=m_1 x + c.$ $\tan \theta_1 = -(1+\sqrt{2}).$ $\theta_1 = \pi - 3\pi/8 = 5\pi/8.$ Yes. Angle between $L_1$ and $L_3$ is $\theta_1-\theta_3 = 5\pi/8 - \pi/8 = 4\pi/8 = \pi/2.$ Correct. 
Angle of $L_6$ is $\theta_6 = -\pi/8.$ 
Let's rotate the coordinate system by $-\pi/8.$ So $L_3$ becomes the new x-axis ( $y'=0$ ). $L_1$ becomes the new y-axis ( $x'=0$ ). 
The angle of $L_6$ in the new system is $\theta_6' = \theta_6 - \theta_3 = -\pi/8 - \pi/8 = -\pi/4.$ 
The line $L_6$ has slope $\tan(-\pi/4) = -1.$ 
The vertex $Z$ is intersection of $L_3$ and $L_6.$ $Z = (-R(\sqrt{2}+1), 0)$ in the original system. $X = (R/2, R(\sqrt{2}+1)/2).$ 
$Z$ is on $L_3.$ $X$ is origin in new system $X'=(0,0).$ $Z'$ must be on the new x-axis. Distance $XZ = \sqrt{R^2 \frac{10+7\sqrt{2}}{2}} = 4+3\sqrt{2}.$ So $Z' = (4+3\sqrt{2}, 0).$ 
The vertex $Y$ is intersection of $L_1$ and $L_6.$ $Y$ is on $L_1.$ $Y'$ must be on the new y-axis. Distance $XY = 4+3\sqrt{2}.$ So $Y' = (0, 4+3\sqrt{2}).$ 
The line $L_6$ passes through $Y'$ and $Z'.$ Equation is $\frac{x'}{4+3\sqrt{2}} + \frac{y'}{4+3\sqrt{2}} = 1.$ $x'+y' = 4+3\sqrt{2}.$ Or $x'+y'-(4+3\sqrt{2})=0.$ 
The distance from $P(x'_p, y'_p)$ to $L_1 (x'=0)$ is $d_1 = |x'_p|.$ 
The distance from $P(x'_p, y'_p)$ to $L_3 (y'=0)$ is $d_3 = |y'_p|.$ 
The distance from $P(x'_p, y'_p)$ to $L_6 (x'+y'-(4+3\sqrt{2})=0)$ is $d_6 = \frac{|x'_p+y'_p-(4+3\sqrt{2})|}{\sqrt{1^2+1^2}} = \frac{|x'_p+y'_p-(4+3\sqrt{2})|}{\sqrt{2}}.$

The formula $\frac{h_1}{\sqrt{2}} + \frac{h_3}{\sqrt{2}} + h_6 = 3+2\sqrt{2}.$ This implies a relation between the heights $h_k.$ As $h_k=A(T_k),$ these are areas. 
Let's assume the formula is about the perpendicular distances $d_k.$ $\frac{d_1}{\sqrt{2}} + \frac{d_3}{\sqrt{2}} + d_6 = C'?$ 
Viviani's theorem states $d_1+d_2+d_3 = H$ for equilateral triangles. For right isosceles triangle $XYZ,$ perhaps $d_1+d_3+\sqrt{2}d_6=H?$ Area of $XYZ = \frac{1}{2} (4+3\sqrt{2})^2.$ The altitude from $X$ to $YZ$ is $h_X = (4+3\sqrt{2})/\sqrt{2}.$

Let's revisit the solution calculation in the provided text again. 
$PW = 1+\sqrt{2}- \frac{1}{\sqrt{2}}\left(\frac{1}{7}+\frac{1}{9}-\frac{1}{4}\right)\pi\left(4+2\sqrt{2}\right)$ was stated in the text. Let's trace this back. 
$PW = 3+2\sqrt{2} - \frac{PU+PV}{\sqrt{2}}.$ 
$PU+PV = (\frac{1}{56}-\frac{1}{72}) A_{circ} + 2(\sqrt{2}+1).$ 
$PW = 3+2\sqrt{2} - \frac{1}{\sqrt{2}} [ (\frac{1}{56}-\frac{1}{72}) A_{circ} + 2(\sqrt{2}+1) ].$ 
$PW = 3+2\sqrt{2} - \frac{A_{circ}}{252\sqrt{2}} - (2+\sqrt{2}) = 1+\sqrt{2} - \frac{A_{circ}}{252\sqrt{2}}.$ This is $A(T_6).$ 
The text continues: $\text{Target Area} = D + PW = D + 1+\sqrt{2} - \frac{A_{circ}}{252\sqrt{2}}.$ 
Target Area $= (\frac{A_{circ}}{8} - (\sqrt{2}+1)) + 1+\sqrt{2} - \frac{A_{circ}}{252\sqrt{2}} = \frac{A_{circ}}{8} - \frac{A_{circ}}{252\sqrt{2}}.$ 
Now the text does some calculation: 
Target Area $= \frac{1}{8} \pi(4+2\sqrt{2}) - (\sqrt{2}+1) + (1+\sqrt{2}) - \frac{1}{\sqrt{2}}(\frac{1}{7}+\frac{1}{9}-\frac{1}{4}) \pi(4+2\sqrt{2}).$ 
This expression should be $D+PW.$ 
The first two terms are $D.$ The next term $(1+\sqrt{2})$ seems to come from $PW.$ Let's check the last term. 
$PW = 1+\sqrt{2} - \frac{A_{circ}}{252\sqrt{2}}.$ 
The last term in the text is $-\frac{1}{\sqrt{2}}(\frac{1}{7}+\frac{1}{9}-\frac{1}{4}) A_{circ} = -\frac{1}{\sqrt{2}}(\frac{16}{63}-\frac{1}{4}) A_{circ} = -\frac{1}{\sqrt{2}}(\frac{64-63}{252}) A_{circ} = -\frac{1}{\sqrt{2}} \frac{1}{252} A_{circ} = -\frac{A_{circ}}{252\sqrt{2}}.$ 
So the expression in the text is correct: $D + PW.$ 
Target Area $= D + 1+\sqrt{2} - \frac{A_{circ}}{252\sqrt{2}}.$ 
Target Area $= \frac{A_{circ}}{8} - (\sqrt{2}+1) + 1+\sqrt{2} - \frac{A_{circ}}{252\sqrt{2}} = \frac{A_{circ}}{8} - \frac{A_{circ}}{252\sqrt{2}}.$ 
This is consistent. 
The text then writes this as $(\frac{1}{8} - \frac{1}{\sqrt{2}}(\frac{1}{7}+\frac{1}{9}-\frac{1}{4})) \text{Total Area}.$ 
This factor is $\frac{1}{8} - \frac{1}{\sqrt{2}}(\frac{1}{252}).$ So $\text{Target Area} = (\frac{1}{8} - \frac{1}{252\sqrt{2}}) A_{circ}.$ 
This is correct. 
Then the text writes $\frac{1}{8}- \frac{\sqrt{2}}{2}\left(\frac{16}{63}-\frac{16}{64}\right).$ Where does this come from? 
$\frac{16}{63}-\frac{16}{64} = 16(\frac{1}{63}-\frac{1}{64}) = 16(\frac{64-63}{63 \times 64}) = 16 \frac{1}{4032} = \frac{1}{252}.$ 
So $\frac{\sqrt{2}}{2}(\frac{1}{252}) = \frac{\sqrt{2}}{504}.$ 
So the expression becomes $\frac{1}{8} - \frac{\sqrt{2}}{504}.$ 
This is the final expression for the scaled area. The area is $1/8 - \sqrt{2}/n.$ So $n=504.$

The calculation in the provided solution seems correct. This means my initial calculation must have an error. Let me check it ONE MORE TIME. 
$A(S_i) = 1/8 - C \cos(\alpha_i - \phi).$ 
$C \cos \psi = -1/56.$ 
$C \sin \psi = -1/72.$ 
$A(S_6) = 1/8 - C \cos(\psi + 5\pi/4).$ 
$\cos(\psi + 5\pi/4) = \cos\psi \cos(5\pi/4) - \sin\psi \sin(5\pi/4).$ 
$\cos(5\pi/4) = -\sqrt{2}/2.$ $\sin(5\pi/4) = -\sqrt{2}/2.$ 
$C \cos(\psi + 5\pi/4) = C [ \cos\psi (-\sqrt{2}/2) - \sin\psi (-\sqrt{2}/2) ] = (-\sqrt{2}/2) [ C \cos\psi + C \sin\psi ].$ 
$= (-\sqrt{2}/2) [ -1/56 - 1/72 ].$ 
$= (-\sqrt{2}/2) [ -(1/56 + 1/72) ].$ 
$1/56 + 1/72 = (9+7)/504 = 16/504 = 2/63.$ 
$= (-\sqrt{2}/2) [-2/63] = \sqrt{2}/63.$ 
$A(S_6) = 1/8 - \sqrt{2}/63.$ 
This result $n=63$ seems consistently derived from my formula.

Is there a flaw in the formula $A(S_i) = 1/8 - C \cos(\alpha_i - \phi)?$ 
$A(S_i) = 1/8 + \mathcal{A}(OPA_i) + \mathcal{A}(OA_{i+1}P).$ 
$A(S_i) = 1/8 + \frac{1}{2} rR (\sin(\theta_i-\phi) - \sin(\theta_{i+1}-\phi)).$ 
Let's recompute the trig part. 
$\sin A - \sin B = 2 \cos \frac{A+B}{2} \sin \frac{A-B}{2}.$ 
$A = \theta_i - \phi.$ $B = \theta_{i+1} - \phi.$ 
$A-B = \theta_i - \theta_{i+1} = -\pi/4.$ $\frac{A-B}{2} = -\pi/8.$ 
$\sin \frac{A-B}{2} = \sin(-\pi/8).$ 
$A+B = \theta_i + \theta_{i+1} - 2\phi.$ $\frac{A+B}{2} = \frac{\theta_i+\theta_{i+1}}{2} - \phi = \alpha_i - \phi.$ 
$A(S_i) = 1/8 + \frac{1}{2} rR (2 \cos(\alpha_i - \phi) \sin(-\pi/8)).$ 
$A(S_i) = 1/8 - rR \sin(\pi/8) \cos(\alpha_i - \phi).$ 
Let $C = rR \sin(\pi/8).$ 
$A(S_i) = 1/8 - C \cos(\alpha_i - \phi).$ 
This formula seems correct.

Let's test the angles again. 
$\alpha_1 = \theta_1 + \pi/8.$ 
$\alpha_3 = \theta_3 + \pi/8 = \theta_1 + 2\pi/4 + \pi/8 = \alpha_1 + \pi/2.$ 
$\alpha_6 = \theta_6 + \pi/8 = \theta_1 + 5\pi/4 + \pi/8 = \alpha_1 + 5\pi/4.$ 
These are correct. 
$C \cos(\alpha_1 - \phi) = -1/56.$ Let $\psi = \alpha_1 - \phi.$ $C \cos \psi = -1/56.$ 
$C \cos(\alpha_3 - \phi) = C \cos(\psi+\pi/2) = - C \sin \psi = 1/72.$ So $C \sin \psi = -1/72.$ 
These are correct. 
We need $A(S_6) = 1/8 - C \cos(\alpha_6 - \phi) = 1/8 - C \cos(\psi+5\pi/4).$ 
$C \cos(\psi+5\pi/4) = C(\cos\psi \cos(5\pi/4) - \sin\psi \sin(5\pi/4))$ 
$= C(\cos\psi (-\sqrt{2}/2) - \sin\psi (-\sqrt{2}/2))$ 
$= (-\sqrt{2}/2) (C \cos\psi + C \sin\psi).$ 
$= (-\sqrt{2}/2) (-1/56 - 1/72).$ 
$= (-\sqrt{2}/2) (-(9+7)/504) = (-\sqrt{2}/2) (-16/504) = (-\sqrt{2}/2) (-2/63).$ 
$= \sqrt{2}/63.$ 
So $A(S_6) = 1/8 - \sqrt{2}/63.$

Maybe the area definition is different? 
"the region bounded by $\overline{PA_1},\overline{PA_2},$ and the minor arc $\widehat{A_1A_2}$ of the circle" 
This is exactly the region $S_1.$ 
My derivation is using the signed area from Green's theorem. Let's think about the orientation. 
$P \to A_i \to A_{i+1} \to P.$ The arc $A_i \to A_{i+1}$ is counterclockwise. $P$ is inside the circle. 
The area is $\int_{PA_i} + \int_{arc A_iA_{i+1}} + \int_{A_{i+1}P}.$ 
$\int_{PA_i} = \mathcal{A}(OPA_i).$ This should be positive if $P$ is to the left of $OA_i$ vector? Let's check orientation. The total area calculated is positive, $1/7$ and $1/9.$ 
Area( $\triangle OPA_i$ ) = $\frac{1}{2} x_p y_i - y_p x_i.$ If $P=(r\cos\phi, r\sin\phi)$ and $A_i=(R\cos\theta_i, R\sin\theta_i),$ this is $\frac{1}{2} r R (\cos\phi \sin\theta_i - \sin\phi \cos\theta_i) = \frac{1}{2} rR \sin(\theta_i-\phi).$ 
Area( $\triangle OA_{i+1}P$ ) = $\frac{1}{2} x_{i+1} y_p - y_{i+1} x_p = \frac{1}{2} R r (\cos\theta_{i+1}\sin\phi - \sin\theta_{i+1}\cos\phi) = \frac{1}{2} Rr \sin(\phi - \theta_{i+1}).$ 
The sum is $A(S_i) = \text{Area(Sector)} + \mathcal{A}(OPA_i) + \mathcal{A}(OA_{i+1}P).$ 
$A(S_i) = 1/8 + \frac{1}{2} rR \sin(\theta_i-\phi) + \frac{1}{2} Rr \sin(\phi - \theta_{i+1}).$ 
$A(S_i) = 1/8 + \frac{1}{2} rR (\sin(\theta_i-\phi) - \sin(\theta_{i+1}-\phi)).$ 
This formula has been checked multiple times.

Let's assume the result $n=504$ is correct. This means my calculation $\sqrt{2}/63$ should be $\sqrt{2}/504.$ 
$\sqrt{2}/63 = 8\sqrt{2}/504.$ 
Where could factor 8 come from? 
Check the sum $1/56 + 1/72 = 16/504.$ This is correct. 
Check the trig $\cos(5\pi/4)=-\sqrt{2}/2, \sin(5\pi/4)=-\sqrt{2}/2.$ Correct. 
Check the sum-product identity. Correct. 
Check the area formula derivation. Correct. 
Check the calculation $C \cos(\psi + 5\pi/4) = (-\sqrt{2}/2) (C \cos\psi + C \sin\psi).$ Correct. 
Check the final calculation $(-\sqrt{2}/2) (-1/56 - 1/72) = (-\sqrt{2}/2) (-16/504) = (\sqrt{2}/2) (16/504) = 8\sqrt{2}/504 = \sqrt{2}/63.$ Correct.

Let's review the alternative solution again. The formula $\frac{PU}{\sqrt{2}} + \frac{PV}{\sqrt{2}} + PW = 3+2\sqrt2$ is suspicious. What if the coefficients are different? Where could $\sqrt{2}$ come from? The triangle $XYZ$ is an isosceles right triangle. Maybe the formula relates distances to the lines in a specific way. Assume $P$ has coordinates $(x', y')$ in the rotated system where $X$ is origin and $L_3, L_1$ are axes. 
$d_3 = |y'|,$ $d_1 = |x'|.$ $d_6 = \frac{|x'+y'-(4+3\sqrt{2})|}{\sqrt{2}}.$ 
Perhaps the formula uses signed distances $d_1', d_3', d_6'$. Assume $P$ is inside $XYZ.$ $x'>0, y'>0.$ $x'+y' < 4+3\sqrt{2}.$ 
$d_1=x', d_3=y'.$ $d_6 = \frac{4+3\sqrt{2}-x'-y'}{\sqrt{2}}.$ 
The formula $\frac{h_1}{\sqrt{2}} + \frac{h_3}{\sqrt{2}} + h_6 = 3+2\sqrt{2}.$ 
Heights $h_k$ are areas $A(T_k).$ 
$h_1 = A(S_1) - D.$ $h_3 = A(S_3) - D.$ $h_6 = A(S_6) - D.$ 
Let's express $A(S_i)$ using my formula. $A(S_i) = 1/8 - K \cos(\alpha_i - \phi),$ where $K=C$ in my notation. 
$h_i = 1/8 - K \cos(\alpha_i - \phi) - D.$ 
$D = 1/8 - A_{triO}.$ So $h_i = A_{triO} - K \cos(\alpha_i - \phi).$ 
$\frac{1}{\sqrt{2}} (A_{triO} - K \cos \psi) + \frac{1}{\sqrt{2}} (A_{triO} - K \cos(\psi+\pi/2)) + (A_{triO} - K \cos(\psi+5\pi/4)) = 3+2\sqrt{2}.$ 
$\frac{A_{triO}}{\sqrt{2}} - \frac{K \cos \psi}{\sqrt{2}} + \frac{A_{triO}}{\sqrt{2}} - \frac{K (-\sin \psi)}{\sqrt{2}} + A_{triO} - K \cos(\psi+5\pi/4) = 3+2\sqrt{2}.$ 
$(\frac{2}{\sqrt{2}}+1) A_{triO} - \frac{K}{\sqrt{2}}(\cos\psi - \sin\psi) - K \cos(\psi+5\pi/4) = 3+2\sqrt{2}.$ 
$A_{triO} = \frac{\sqrt{2}+1}{\pi(4+2\sqrt{2})}.$ This value is defined relative to Area=1 circle. The alternative solution uses $A_{circ}=\pi(4+2\sqrt{2}).$ So $A_{triO}$ there is $\sqrt{2}+1.$ 
$K=C=rR \sin(\pi/8).$ In the alternative solution units, $R^2=4+2\sqrt{2}.$ $R=\sqrt{4+2\sqrt{2}}.$ $C = r \sqrt{4+2\sqrt{2}} \sin(\pi/8).$ 
$C \cos \psi = - A_{circ}/56.$ $C \sin \psi = - A_{circ}/72.$ 
The relation is supposed to hold true for areas $A(T_k).$ Let's use the alternative solution values for $h_k.$ 
$h_1 = \frac{A_{circ}}{56} + \sqrt{2}+1.$ $h_3 = -\frac{A_{circ}}{72} + \sqrt{2}+1.$ $h_6 = A(S_6) - \frac{A_{circ}}{8} + \sqrt{2}+1.$ 
$\frac{1}{\sqrt{2}} (\frac{A_{circ}}{56} + \sqrt{2}+1) + \frac{1}{\sqrt{2}} (-\frac{A_{circ}}{72} + \sqrt{2}+1) + h_6 = 3+2\sqrt{2}.$ 
$\frac{A_{circ}}{\sqrt{2}} (\frac{1}{56}-\frac{1}{72}) + \frac{1}{\sqrt{2}}(2\sqrt{2}+2) + h_6 = 3+2\sqrt{2}.$ 
$\frac{A_{circ}}{252\sqrt{2}} + 2+\sqrt{2} + h_6 = 3+2\sqrt{2}.$ 
$h_6 = 1+\sqrt{2} - \frac{A_{circ}}{252\sqrt{2}}.$ This step is verified.

Let's assume the relation $\frac{h_1}{\sqrt{2}} + \frac{h_3}{\sqrt{2}} + h_6 = 3+2\sqrt{2}$ is correct. The derivation $n=504$ follows. 
Let's question my initial derivation $A(S_i) = 1/8 - C \cos(\alpha_i - \phi).$ 
Suppose $P=(x,y)$ in Cartesian coordinate system with $O=(0,0).$ Let $A_i=(R\cos\theta_i, R\sin\theta_i).$ 
Area( $S_i$ ) = Area( $\triangle PA_iA_{i+1}$ ) + Area(Segment $A_iA_{i+1}$ ). 
Area( $\triangle PA_iA_{i+1}) = \frac{1}{2} | \vec{PA_i} \times \vec{PA_{i+1}} |.$ 
$\vec{PA_i} = (R\cos\theta_i - x, R\sin\theta_i - y).$ 
$\vec{PA_{i+1}} = (R\cos\theta_{i+1} - x, R\sin\theta_{i+1} - y).$ 
Area( $\triangle PA_iA_{i+1}) = \frac{1}{2} | (R\cos\theta_i - x)(R\sin\theta_{i+1} - y) - (R\sin\theta_i - y)(R\cos\theta_{i+1} - x) |.$ 
$= \frac{1}{2} | R^2 \cos\theta_i \sin\theta_{i+1} - R y \cos\theta_i - R x \sin\theta_{i+1} + xy - (R^2 \sin\theta_i \cos\theta_{i+1} - R x \sin\theta_i - R y \cos\theta_{i+1} + xy) |.$ 
$= \frac{1}{2} | R^2 \sin(\theta_{i+1}-\theta_i) - R y (\cos\theta_i - \cos\theta_{i+1}) + R x (\sin\theta_i - \sin\theta_{i+1}) |.$ 
$= \frac{1}{2} | R^2 \sin(\pi/4) - Ry (2\sin(\frac{\theta_i+\theta_{i+1}}{2})\sin(\frac{\theta_{i+1}-\theta_i}{2})) + Rx ( -2\cos(\frac{\theta_i+\theta_{i+1}}{2})\sin(\frac{\theta_{i+1}-\theta_i}{2})) |.$ 
$\theta_{i+1}-\theta_i = \pi/4.$ $\frac{\theta_{i+1}-\theta_i}{2}=\pi/8.$ $\frac{\theta_i+\theta_{i+1}}{2} = \alpha_i.$ 
$= \frac{1}{2} | R^2 \sin(\pi/4) - 2 Ry \sin\alpha_i \sin(\pi/8) - 2 Rx \cos\alpha_i \sin(\pi/8) |.$ 
$= \frac{1}{2} | R^2 \frac{\sqrt{2}}{2} - 2 R \sin(\pi/8) (y \sin\alpha_i + x \cos\alpha_i) |.$ 
Let $P = (r\cos\phi, r\sin\phi).$ $x=r\cos\phi, y=r\sin\phi.$ 
$y \sin\alpha_i + x \cos\alpha_i = r\sin\phi \sin\alpha_i + r\cos\phi \cos\alpha_i = r \cos(\alpha_i - \phi).$ 
$= \frac{1}{2} | R^2 \frac{\sqrt{2}}{2} - 2 R r \sin(\pi/8) \cos(\alpha_i - \phi) |.$ 
Area( $\triangle OA_iA_{i+1}) = \frac{1}{2} R^2 \sin(\pi/4) = \frac{1}{2} R^2 \frac{\sqrt{2}}{2}.$ 
$A(T_i) = | A_{triO} - r R \sin(\pi/8) \cos(\alpha_i - \phi) |.$ 
$A(S_i) = A(T_i) + D = A(T_i) + A_{sec} - A_{triO}.$ 
If $A_{triO} - r R \sin(\pi/8) \cos(\alpha_i - \phi)$ is positive, then $A(S_i) = A_{triO} - r R \sin(\pi/8) \cos(\alpha_i - \phi) + A_{sec} - A_{triO}.$ 
$A(S_i) = A_{sec} - r R \sin(\pi/8) \cos(\alpha_i - \phi).$ 
This matches my formula $A(S_i) = 1/8 - C \cos(\alpha_i - \phi).$ 
The term $A_{triO} - C \cos(\alpha_i - \phi)$ must be positive. Is it always? $A_{triO} = \frac{R^2\sqrt{2}}{4}.$ $C = rR \sin(\pi/8).$ 
Condition is $\frac{R^2\sqrt{2}}{4} > rR \sin(\pi/8) \cos(\alpha_i - \phi).$ 
Since $r<R,$ we need $\frac{R\sqrt{2}}{4} > \sin(\pi/8) \cos(\alpha_i - \phi).$ Max value of RHS is $\sin(\pi/8).$ 
Is $\frac{R\sqrt{2}}{4} > \sin(\pi/8)?$ $\sin(\pi/8) = \frac{\sqrt{2-\sqrt{2}}}{2}.$ $R=1/\sqrt{\pi}.$ $\frac{1/\sqrt{\pi} \sqrt{2}}{4} = \frac{\sqrt{2}}{4\sqrt{\pi}}.$ We compare $\frac{\sqrt{2}}{4\sqrt{\pi}}$ and $\frac{\sqrt{2-\sqrt{2}}}{2}.$ 
$(\frac{\sqrt{2}}{4\sqrt{\pi}})^2 = \frac{2}{16\pi} = \frac{1}{8\pi}.$ $(\frac{\sqrt{2-\sqrt{2}}}{2})^2 = \frac{2-\sqrt{2}}{4}.$ 
Is $\frac{1}{8\pi} > \frac{2-\sqrt{2}}{4}?$ $1 > 2\pi(2-\sqrt{2}).$ $1 > 4\pi - 2\pi\sqrt{2}.$ $\pi \approx 3.14.$ $4\pi \approx 12.56.$ $2\pi\sqrt{2} \approx 6.28 \times 1.414 \approx 8.88.$ $1 > 12.56 - 8.88 = 3.68.$ This is false. 
So the absolute value is important. Area( $T_i$ ) must be positive. 
$A(T_i) = \frac{1}{2} s h_i,$ so area must be positive. 
$A(S_1) = 1/7 = 1/8 - C \cos \psi.$ $C \cos \psi = -1/56.$ $A(T_1) = A_{triO} - C \cos \psi = A_{triO} + 1/56.$ Positive. 
$A(S_3) = 1/9 = 1/8 - C \cos(\psi+\pi/2) = 1/8 + C \sin \psi.$ $C \sin \psi = -1/72.$ $A(T_3) = A_{triO} - C \cos(\psi+\pi/2) = A_{triO} + C \sin \psi = A_{triO} - 1/72.$ Is this positive? $A_{triO} = \frac{\text{Area}(\triangle OA_1A_2)}{\text{Area(Circle)}} = \frac{R^2\sqrt{2}/4}{\pi R^2} = \frac{\sqrt{2}}{4\pi}.$ $\sqrt{2}/(4\pi) \approx 1.414 / (4 \times 3.14) \approx 1.414 / 12.56 \approx 0.112.$ $1/72 \approx 0.0138.$ It is positive. 
$A(S_6) = 1/8 - C \cos(\psi+5\pi/4).$ $A(T_6) = A_{triO} - C \cos(\psi+5\pi/4).$ We found $C \cos(\psi+5\pi/4) = \sqrt{2}/63.$ 
$A(T_6) = A_{triO} - \sqrt{2}/63.$ Is this positive? $A_{triO} = \sqrt{2}/(4\pi).$ $\sqrt{2}/(4\pi) - \sqrt{2}/63 = \sqrt{2} (1/(4\pi) - 1/63).$ $1/(4\pi) \approx 1/12.56 \approx 0.079.$ $1/63 \approx 0.0158.$ Yes, positive. 
So the absolute value is not an issue. My derivation holds.

Let's check the mysterious formula $\frac{h_1}{\sqrt{2}} + \frac{h_3}{\sqrt{2}} + h_6 = 3+2\sqrt{2}.$ 
The problem is set up with area 1. The alternative solution rescales it. 
Let's use my notation. $A(S_i)$ is the area given in the problem. $A_{circ}=1.$ 
$A(T_i) = A(S_i) - (1/8 - A_{triO}).$ 
$A_{triO} = \sqrt{2}/(4\pi).$ 
$h_1 = A(T_1) = 1/7 - (1/8 - A_{triO}) = 1/56 + A_{triO}.$ 
$h_3 = A(T_3) = 1/9 - (1/8 - A_{triO}) = -1/72 + A_{triO}.$ 
$h_6 = A(T_6) = A(S_6) - (1/8 - A_{triO}).$ 
The formula uses $h_i$ values based on $s=2,$ which means $A_{circ}=\pi(4+2\sqrt{2}).$ Let's call these $h_i'.$ 
$h_1' = A(T_1') = \frac{A_{circ}}{56} + A_{triO}' = \frac{\pi(4+2\sqrt{2})}{56} + \sqrt{2}+1.$ 
$h_3' = A(T_3') = -\frac{A_{circ}}{72} + A_{triO}' = -\frac{\pi(4+2\sqrt{2})}{72} + \sqrt{2}+1.$ 
$h_6' = A(T_6') = A(S_6') - D' = A(S_6') - (\frac{A_{circ}}{8} - (\sqrt{2}+1)).$ 
The formula is $\frac{h_1'}{\sqrt{2}} + \frac{h_3'}{\sqrt{2}} + h_6' = 3+2\sqrt{2}.$ 
The result was $A(S_6') = \frac{A_{circ}}{8} - \frac{A_{circ}}{252\sqrt{2}}.$ 
Scaling back to $A_{circ}=1$: $A(S_6) = 1/8 - \frac{1}{252\sqrt{2}} = 1/8 - \frac{\sqrt{2}}{504}.$ $n=504.$

Let's check the formula $\frac{d_1}{\sqrt{2}} + \frac{d_3}{\sqrt{2}} + d_6 = \text{constant}.$ For perpendicular distances $d_i.$ 
For $P=O,$ $O=(0,0).$ Distances to lines $L_1, L_3, L_6$ are all $h=\sqrt{2}+1.$ (In units of $s=2$ ). 
$\frac{\sqrt{2}+1}{\sqrt{2}} + \frac{\sqrt{2}+1}{\sqrt{2}} + (\sqrt{2}+1) = \frac{2(\sqrt{2}+1)}{\sqrt{2}} + \sqrt{2}+1 = \sqrt{2}(\sqrt{2}+1) + \sqrt{2}+1 = (2+\sqrt{2}) + (\sqrt{2}+1) = 3+2\sqrt{2}.$ 
This works for $P=O$ if $h_k$ are perpendicular distances $d_k.$ 
But $h_k$ are heights of the triangles $PA_kA_{k+1}.$ These are not generally equal to $d_k.$ 
Height $h_k = A(T_k).$ Area $A(T_k) = |\frac{1}{2} R^2 \sin(\pi/4) - R r \sin(\pi/8) \cos(\alpha_i - \phi)|.$ Using $A_{circ}=1.$ 
$A(T_k) = | A_{triO} - C \cos(\alpha_k - \phi) |.$ 
$d_k = \frac{|L_k(x,y)|}{\sqrt{a_k^2+b_k^2}}.$ 
Let $P$ be near $A_1.$ Then height $h_1$ tends to 0. Distance $d_1$ also tends to 0.

Is it possible that the area formula $A(S_i)$ depends linearly on the coordinates of $P?$ 
$A(S_i) = 1/8 + \frac{1}{2} x_p (y_i - y_{i+1}) + \frac{1}{2} y_p (x_{i+1} - x_i).$ 
Let $P=(x,y).$ 
$A(S_i) = 1/8 + \frac{1}{2} x (R \sin\theta_i - R \sin\theta_{i+1}) + \frac{1}{2} y (R \cos\theta_{i+1} - R \cos\theta_i).$ 
$A(S_i) = 1/8 + \frac{R}{2} x (\sin\theta_i - \sin\theta_{i+1}) + \frac{R}{2} y (\cos\theta_{i+1} - \cos\theta_i).$ 
$\sin\theta_i - \sin\theta_{i+1} = 2 \cos(\alpha_i) \sin(-\pi/8) = -2 \cos\alpha_i \sin(\pi/8).$ 
$\cos\theta_{i+1} - \cos\theta_i = -2 \sin\alpha_i \sin(\pi/8).$ 
$A(S_i) = 1/8 + \frac{R}{2} x (-2 \cos\alpha_i \sin(\pi/8)) + \frac{R}{2} y (-2 \sin\alpha_i \sin(\pi/8)).$ 
$A(S_i) = 1/8 - R \sin(\pi/8) [ x \cos\alpha_i + y \sin\alpha_i ].$ 
Let $P = (r\cos\phi, r\sin\phi).$ 
$A(S_i) = 1/8 - R \sin(\pi/8) [ r \cos\phi \cos\alpha_i + r \sin\phi \sin\alpha_i ].$ 
$A(S_i) = 1/8 - r R \sin(\pi/8) \cos(\alpha_i - \phi).$ This confirms my formula again. 
$A(S_i)$ is a linear function of $x$ and $y.$ $A(S_i) = 1/8 - C_i x - D_i y.$ 
$C_i = R \sin(\pi/8) \cos\alpha_i.$ $D_i = R \sin(\pi/8) \sin\alpha_i.$ 
The area $A(T_i) = A(S_i) - D = A(S_i) - (1/8 - A_{triO}) = A_{triO} - C_i x - D_i y.$ 
$h_i$ is $A(T_i)$ in the $s=2$ scale. $A(T_i') = A_{triO}' - C_i' x - D_i' y.$ 
$A_{triO}'=\sqrt{2}+1.$ $R' = \sqrt{4+2\sqrt{2}}.$ 
$C_i' = R' \sin(\pi/8) \cos\alpha_i.$ $D_i' = R' \sin(\pi/8) \sin\alpha_i.$ 
The formula $\frac{h_1'}{\sqrt{2}} + \frac{h_3'}{\sqrt{2}} + h_6' = 3+2\sqrt{2}.$ 
Let's check if this holds for $P=(x,y).$ 
$LHS = \frac{1}{\sqrt{2}} (A_{triO}' - C_1' x - D_1' y) + \frac{1}{\sqrt{2}} (A_{triO}' - C_3' x - D_3' y) + (A_{triO}' - C_6' x - D_6' y).$ 
$= (\frac{2}{\sqrt{2}}+1) A_{triO}' - x (\frac{C_1'}{\sqrt{2}} + \frac{C_3'}{\sqrt{2}} + C_6') - y (\frac{D_1'}{\sqrt{2}} + \frac{D_3'}{\sqrt{2}} + D_6').$ 
$= (\sqrt{2}+1) A_{triO}' - x R' \sin(\pi/8) (\frac{\cos\alpha_1}{\sqrt{2}} + \frac{\cos\alpha_3}{\sqrt{2}} + \cos\alpha_6) - y R' \sin(\pi/8) (\frac{\sin\alpha_1}{\sqrt{2}} + \frac{\sin\alpha_3}{\sqrt{2}} + \sin\alpha_6).$ 
$(\sqrt{2}+1) A_{triO}' = (\sqrt{2}+1)(\sqrt{2}+1) = 3+2\sqrt{2}.$ 
So the formula holds if the coefficients of $x$ and $y$ are zero. 
Need to check: 
$\frac{\cos\alpha_1}{\sqrt{2}} + \frac{\cos\alpha_3}{\sqrt{2}} + \cos\alpha_6 = 0.$ 
$\frac{\sin\alpha_1}{\sqrt{2}} + \frac{\sin\alpha_3}{\sqrt{2}} + \sin\alpha_6 = 0.$ 
$\alpha_3 = \alpha_1 + \pi/2.$ $\cos\alpha_3 = -\sin\alpha_1.$ $\sin\alpha_3 = \cos\alpha_1.$ 
$\alpha_6 = \alpha_1 + 5\pi/4.$ $\cos\alpha_6 = \cos(\alpha_1+5\pi/4) = \cos\alpha_1 \cos(5\pi/4) - \sin\alpha_1 \sin(5\pi/4) = \cos\alpha_1 (-\sqrt{2}/2) - \sin\alpha_1 (-\sqrt{2}/2) = (-\sqrt{2}/2)(\cos\alpha_1 - \sin\alpha_1).$ 
$\sin\alpha_6 = \sin(\alpha_1+5\pi/4) = \sin\alpha_1 \cos(5\pi/4) + \cos\alpha_1 \sin(5\pi/4) = \sin\alpha_1 (-\sqrt{2}/2) + \cos\alpha_1 (-\sqrt{2}/2) = (-\sqrt{2}/2)(\sin\alpha_1 + \cos\alpha_1).$

Check the first sum: 
$\frac{\cos\alpha_1}{\sqrt{2}} + \frac{-\sin\alpha_1}{\sqrt{2}} + (-\frac{\sqrt{2}}{2})(\cos\alpha_1 - \sin\alpha_1).$ 
$= \frac{1}{\sqrt{2}}(\cos\alpha_1 - \sin\alpha_1) - \frac{1}{\sqrt{2}}(\cos\alpha_1 - \sin\alpha_1) = 0.$ It works.

Check the second sum: 
$\frac{\sin\alpha_1}{\sqrt{2}} + \frac{\cos\alpha_1}{\sqrt{2}} + (-\frac{\sqrt{2}}{2})(\sin\alpha_1 + \cos\alpha_1).$ 
$= \frac{1}{\sqrt{2}}(\sin\alpha_1 + \cos\alpha_1) - \frac{1}{\sqrt{2}}(\sin\alpha_1 + \cos\alpha_1) = 0.$ It also works.

So the mysterious formula is correct. It holds for any point $P(x,y).$ 
Therefore, the derivation leading to $n=504$ is correct.

My original calculation must be wrong somewhere. 
$A(S_6) = 1/8 - C \cos(\psi + 5\pi/4).$ 
$C \cos(\psi+5\pi/4) = (-\sqrt{2}/2) (C \cos\psi + C \sin\psi).$ 
$= (-\sqrt{2}/2) (-1/56 - 1/72) = (-\sqrt{2}/2) (- (9+7)/504) = (-\sqrt{2}/2) (-16/504).$ 
$= (\sqrt{2}/2) (16/504) = 8\sqrt{2}/504 = \sqrt{2}/63.$ 
Where did I make a mistake? 
Ah, $16/504 = 2/63.$ Correct. 
$(\sqrt{2}/2) (16/504) = \sqrt{2}/2 \times (2/63) = \sqrt{2}/63.$ This calculation IS correct.

Let's review the relation $A(S_i) = 1/8 - C \cos(\alpha_i - \phi).$ 
$A(S_1) = 1/7.$ $1/7 = 1/8 - C \cos(\alpha_1 - \phi).$ $C \cos(\alpha_1 - \phi) = 1/8 - 1/7 = -1/56.$ 
$A(S_3) = 1/9.$ $1/9 = 1/8 - C \cos(\alpha_3 - \phi).$ $C \cos(\alpha_3 - \phi) = 1/8 - 1/9 = 1/72.$ 
$A(S_6) = 1/8 - C \cos(\alpha_6 - \phi).$ 
$\alpha_3 = \alpha_1 + \pi/2.$ $C \cos(\alpha_1 - \phi + \pi/2) = - C \sin(\alpha_1 - \phi) = 1/72.$ $C \sin(\alpha_1 - \phi) = -1/72.$ 
$\alpha_6 = \alpha_1 + 5\pi/4.$ $C \cos(\alpha_1 - \phi + 5\pi/4) = C \cos(\psi + 5\pi/4).$ 
$= C (\cos\psi \cos(5\pi/4) - \sin\psi \sin(5\pi/4)).$ 
$= C\cos\psi (-\sqrt{2}/2) - C\sin\psi (-\sqrt{2}/2).$ 
$= (-1/56)(-\sqrt{2}/2) - (-1/72)(-\sqrt{2}/2).$ 
$= \frac{\sqrt{2}}{112} - \frac{\sqrt{2}}{144}.$ 
$= \sqrt{2} (\frac{1}{112} - \frac{1}{144}).$ $112 = 16 \times 7.$ $144 = 16 \times 9.$ 
$= \sqrt{2} (\frac{1}{16 \times 7} - \frac{1}{16 \times 9}) = \frac{\sqrt{2}}{16} (\frac{1}{7} - \frac{1}{9}).$ 
$= \frac{\sqrt{2}}{16} (\frac{9-7}{63}) = \frac{\sqrt{2}}{16} (\frac{2}{63}) = \frac{\sqrt{2}}{8 \times 63} = \frac{\sqrt{2}}{504}.$ 
So $C \cos(\alpha_6 - \phi) = \frac{\sqrt{2}}{504}.$ 
$A(S_6) = 1/8 - \frac{\sqrt{2}}{504}.$ 
Therefore $n=504.$

Let's see what mistake I made in the previous calculation. 
$C \cos(\psi+5\pi/4) = (-\sqrt{2}/2) (C \cos\psi + C \sin\psi).$ 
$= (-\sqrt{2}/2) (-1/56 + (-1/72)) = (-\sqrt{2}/2) (-1/56 - 1/72).$ Yes, this step is correct. 
$= (-\sqrt{2}/2) (-(1/56 + 1/72)).$ Yes, correct. 
$1/56 + 1/72 = 16/504 = 2/63.$ Yes, correct. 
$= (-\sqrt{2}/2) (-2/63).$ Yes, correct. 
$= (\sqrt{2}/2) (2/63) = \sqrt{2}/63.$ Yes, correct. 
Wait, $(\sqrt{2}/2) (2/63) = \sqrt{2}/63.$ This calculation is correct. Why did the individual substitution give $\sqrt{2}/504$? 
Let's re-calculate: 
$C\cos\psi (-\sqrt{2}/2) - C\sin\psi (-\sqrt{2}/2) = (-\frac{\sqrt{2}}{2}) C\cos\psi + (\frac{\sqrt{2}}{2}) C\sin\psi.$ 
$= (-\frac{\sqrt{2}}{2}) (-1/56) + (\frac{\sqrt{2}}{2}) (-1/72).$ 
$= \frac{\sqrt{2}}{112} - \frac{\sqrt{2}}{144}.$ This is what I got just now. 
$= \sqrt{2}(\frac{1}{112} - \frac{1}{144}) = \sqrt{2}(\frac{9-7}{1008}) = \sqrt{2}(\frac{2}{1008}) = \frac{2\sqrt{2}}{1008} = \frac{\sqrt{2}}{504}.$ 
OK, $16 \times 7 \times 9 = 16 \times 63 = 1008.$ LCM(112, 144) = LCM(16x7, 16x9) = 16x63 = 1008. 
$\frac{1}{112} - \frac{1}{144} = \frac{9}{1008} - \frac{7}{1008} = \frac{2}{1008}.$ Correct.

Where is the error in the $(-\sqrt{2}/2) (C \cos\psi + C \sin\psi)$ calculation? 
$C \cos\psi + C \sin\psi = -1/56 - 1/72 = -(1/56+1/72) = -( (9+7)/504 ) = -16/504 = -2/63.$ Correct. 
$C \cos(\psi+5\pi/4) = (-\sqrt{2}/2) (C \cos\psi + C \sin\psi).$ 
$= (-\sqrt{2}/2) (-2/63).$ Correct. 
$= \frac{2\sqrt{2}}{2 \times 63} = \frac{\sqrt{2}}{63}.$ Correct.

So where is the discrepancy? 
$C \cos(\psi+5\pi/4) = C(\cos\psi \cos(5\pi/4) - \sin\psi \sin(5\pi/4))$ 
$= C\cos\psi (-\sqrt{2}/2) - C\sin\psi (-\sqrt{2}/2)$ 
$= (-\sqrt{2}/2) C\cos\psi + (\sqrt{2}/2) C\sin\psi.$ 
This is NOT equal to $(-\sqrt{2}/2) (C \cos\psi + C \sin\psi).$ 
Ah, I made a sign error in the factorization. 
$\cos(\psi+5\pi/4) = -\frac{\sqrt{2}}{2}\cos\psi + \frac{\sqrt{2}}{2}\sin\psi = (-\frac{\sqrt{2}}{2}) (\cos\psi - \sin\psi).$ 
Let's calculate $C \cos(\psi+5\pi/4)$ using this correct factorization. 
$= (-\sqrt{2}/2) (C\cos\psi - C\sin\psi).$ 
$= (-\sqrt{2}/2) (-1/56 - (-1/72)).$ 
$= (-\sqrt{2}/2) (-1/56 + 1/72).$ 
$= (-\sqrt{2}/2) ((-9+7)/504).$ 
$= (-\sqrt{2}/2) (-2/504).$ 
$= (\sqrt{2}/2) (2/504) = \sqrt{2}/504.$ 
Okay, finally found my mistake. It was a sign error in the trigonometric calculation. 
So $A(S_6) = 1/8 - \sqrt{2}/504.$ 
$n=504.$

The alternative solution was indeed correct, and its derivation using the linear property of the area function is sound. My initial calculation had a sign error.